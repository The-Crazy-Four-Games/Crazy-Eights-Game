\documentclass{article}


\usepackage[T1]{fontenc}
\usepackage[utf8]{inputenc}
\usepackage[english]{babel}

\usepackage{tabularx,booktabs,array,ragged2e,makecell}
\usepackage[none]{hyphenat}
\renewcommand\theadfont{\normalsize\bfseries}
\renewcommand{\arraystretch}{1.2}
\setlength{\tabcolsep}{6pt}
\usepackage{float}



\newcolumntype{L}{>{\RaggedRight\arraybackslash}X}

\title{Problem Statement and Goals\\\progname}
\author{\authname}
\date{}

%% Comments

\usepackage{color}

\newif\ifcomments\commentstrue %displays comments
%\newif\ifcomments\commentsfalse %so that comments do not display

\ifcomments
\newcommand{\authornote}[3]{\textcolor{#1}{[#3 ---#2]}}
\newcommand{\todo}[1]{\textcolor{red}{[TODO: #1]}}
\else
\newcommand{\authornote}[3]{}
\newcommand{\todo}[1]{}
\fi

\newcommand{\wss}[1]{\authornote{magenta}{SS}{#1}} 
\newcommand{\plt}[1]{\authornote{cyan}{TPLT}{#1}} %For explanation of the template
\newcommand{\an}[1]{\authornote{cyan}{Author}{#1}}

%% Common Parts

\newcommand{\progname}{ProgName} % PUT YOUR PROGRAM NAME HERE
\newcommand{\authname}{Team \#, Team Name
\\ Student 1 name
\\ Student 2 name
\\ Student 3 name
\\ Student 4 name} % AUTHOR NAMES                  

\usepackage{hyperref}
    \hypersetup{colorlinks=true, linkcolor=blue, citecolor=blue, filecolor=blue,
                urlcolor=blue, unicode=false}
    \urlstyle{same}
                                



\begin{document}
    \maketitle

    \begin{table}[hp]
        \caption{Revision History} \label{TblRevisionHistory}
        \begin{tabularx}{\textwidth}{llX}
            \toprule
            \textbf{Date} & \textbf{Developer(s)} & \textbf{Change}\\
            \midrule
            Date1 & Name(s) & Description of changes\\
            Date2 & Name(s) & Description of changes\\
            ... & ... & ...\\
            \bottomrule
        \end{tabularx}
    \end{table}

    \section{Problem Statement}

    \subsection{Problem}


    \subsection{Inputs and Outputs}


    \subsection{Stakeholders}


    \subsection{Environment}

    \section{Goals}

    \subsection{Minimum Viable Product (MVP) Goals}

    \subsubsection{Functional Goals}
    \begin{table}[H]
        \caption{MVP Functional Goals}\label{TblMVPFunctionalGoals}
        \begin{tabular}{|p{0.22\textwidth}|p{0.46\textwidth}|p{0.28\textwidth}|}
            \hline
            \textbf{Goal} & \textbf{Explanation} & \textbf{Reasoning} \\
            \hline
            Two-Player Core Loop &
            Support a one-versus-one match with turn-taking, drawing, discarding, and win checks. Includes starter card, discard pile, and reshuffling the stock pile. &
            Two players are the smallest playable unit; finishing this proves the core loop. \\
            \hline
            Classic Rules Engine &
            Implement standard Crazy Eights: match by suit or rank; an ``8'' is wild and lets the player declare a suit; draw if no valid move exists. &
            Correct classic rules establish a solid baseline before adding variations. \\
            \hline
            Dozenal (Base-12) Scoring/Display &
            Display scores, counters, or thresholds in base-12 notation while keeping classic rules unchanged. &
            Introduces dozenal in a simple, non-disruptive way that highlights novelty while retaining accessibility. \\
            \hline
        \end{tabular}
    \end{table}

    \subsubsection{Non-functional Goals}
    \begin{table}[H]
        \caption{MVP Non-functional Goals}\label{TblMVPNonFunctionalGoals}
        \begin{tabular}{|p{0.22\textwidth}|p{0.46\textwidth}|p{0.28\textwidth}|}
            \hline
            \textbf{Goal} & \textbf{Explanation} & \textbf{Reasoning} \\
            \hline
            Move Validation and Feedback &
            Provide immediate invalid-move feedback, clear suit selection UI after playing an ``8,'' and real-time state indicators. &
            Reduces errors and learning curve; improves usability. \\
            \hline
            Testability and Determinism &
            Support seeded shuffling and provide basic logs or replays for each session. &
            Enables reproducible unit/integration testing and debugging. \\
            \hline
            Stability and Performance &
            Ensure responsive UI \(<200\,\text{ms}\), no crashes, no deadlocks, and correct reshuffling. &
            Reliability is the baseline for acceptance and live demonstration. \\
            \hline
            Minimal UI &
            Provide a desktop~or web interface showing hand, discard pile, current state, and a dozenal score tracker. &
            Covers essential interactions while limiting complexity at the MVP stage. \\
            \hline
        \end{tabular}
    \end{table}

    \subsubsection{Non-functional Goals}
    \begin{table}[h!]
        \caption{MVP Non-functional Goals}\label{TblMVPNonFunctionalGoals}
        \begin{tabularx}{\textwidth}{|L|L|L|}
            \hline
            \textbf{Goal} & \textbf{Explanation} & \textbf{Reasoning} \\
            \hline
            Move Validation and Feedback &
            Provide immediate invalid-move feedback, clear suit selection UI after playing an ``8,'' and real-time state indicators. &
            Reduces errors and learning curve; improves usability. \\
            \hline
            Testability and Determinism &
            Support seeded shuffling and provide basic logs or replays for each session. &
            Enables reproducible unit/integration testing and debugging. \\
            \hline
            Stability and Performance &
            Ensure responsive UI, no crashes, no deadlocks, and correct reshuffling. &
            Reliability is the baseline for acceptance and live demonstration. \\
            \hline
            Minimal UI &
            Provide a desktop~or web interface showing hand, discard pile, current state, and a dozenal score tracker. &
            Covers essential interactions while limiting complexity at the MVP stage. \\
            \hline
        \end{tabularx}
    \end{table}

    \section{Stretch Goals}

    \subsection{Functional Stretch Goals}
    \begin{table}[H]
        \caption{Functional Stretch Goals}\label{TblStretchFunctionalGoals}
        \begin{tabular}{|p{0.22\textwidth}|p{0.46\textwidth}|p{0.28\textwidth}|}
            \hline
            \textbf{Goal} & \textbf{Explanation} & \textbf{Reasoning} \\
            \hline
            3--4 Player Matches &
            Extend gameplay to three or more players with clear turn rotation and visualization. &
            Increases replayability; closer to common play. \\
            \hline
            Online Multiplayer &
            Create/join rooms and synchronize game state; provide basic reconnection. &
            Demonstrates system design capability and supports real usage. \\
            \hline
            AI Opponent &
            Computer-controlled players with simple heuristics for card choice and suit declaration. &
            Enables single-player testing and showcases extensibility. \\
            \hline
            Advanced Dozenal Variants &
            Optional rule packs (e.g., effects for 12-related cards; base-12 scoring thresholds such as \(60_{12}\)). &
            Deepens the dozenal theme while keeping the classic mode intact. \\
            \hline
            Rule Configurator &
            Toggles for house rules (draw-until-playable, stacking eights, scoring methods in decimal or dozenal). &
            Shows variability management; supports experimentation and future product-line thinking. \\
            \hline
        \end{tabular}
    \end{table}

    \subsection{Non-functional Stretch Goals}
    \begin{table}[H]
        \caption{Non-functional Stretch Goals}\label{TblStretchNonFunctionalGoals}
        \begin{tabular}{|p{0.22\textwidth}|p{0.46\textwidth}|p{0.28\textwidth}|}
            \hline
            \textbf{Goal} & \textbf{Explanation} & \textbf{Reasoning} \\
            \hline
            Tutorial and Visual Guidance &
            First-game tutorial, invalid-move highlights, and play suggestions. &
            Lowers the learning curve and improves usability. \\
            \hline
            Save/Replay System &
            Save completed games and replay event sequences. &
            Aids debugging, user study, and documentation. \\
            \hline
            Cross-Platform Packaging &
            Deploy as a web app or desktop executable (Windows/macOS). &
            Lowers the barrier for evaluators and external users to try the system. \\
            \hline
        \end{tabular}
    \end{table}

    \section{Extras}


    \newpage{}
    \section*{Appendix --- Reflection}

    The purpose of reflection questions is to give you a chance to assess your own
learning and that of your group as a whole, and to find ways to improve in the
future. Reflection is an important part of the learning process.  Reflection is
also an essential component of a successful software development process.  

Reflections are most interesting and useful when they're honest, even if the
stories they tell are imperfect. You will be marked based on your depth of
thought and analysis, and not based on the content of the reflections
themselves. Thus, for full marks we encourage you to answer openly and honestly
and to avoid simply writing ``what you think the evaluator wants to hear.''

Please answer the following questions.  Some questions can be answered on the
team level, but where appropriate, each team member should write their own
response:


    \begin{enumerate}
        \item What went well while writing this deliverable?
        \item What pain points did you experience during this deliverable, and how
        did you resolve them?
        \item How did you and your team adjust the scope of your goals to ensure
        they are suitable for a Capstone project (not overly ambitious but also of
        appropriate complexity for a senior design project)?
    \end{enumerate}

\end{document}