\documentclass{article}

\usepackage{tabularx}
\usepackage{booktabs}

\title{Problem Statement and Goals\\\progname}

\author{\authname}

\date{}

%% Comments

\usepackage{color}

\newif\ifcomments\commentstrue %displays comments
%\newif\ifcomments\commentsfalse %so that comments do not display

\ifcomments
\newcommand{\authornote}[3]{\textcolor{#1}{[#3 ---#2]}}
\newcommand{\todo}[1]{\textcolor{red}{[TODO: #1]}}
\else
\newcommand{\authornote}[3]{}
\newcommand{\todo}[1]{}
\fi

\newcommand{\wss}[1]{\authornote{magenta}{SS}{#1}} 
\newcommand{\plt}[1]{\authornote{cyan}{TPLT}{#1}} %For explanation of the template
\newcommand{\an}[1]{\authornote{cyan}{Author}{#1}}

%% Common Parts

\newcommand{\progname}{ProgName} % PUT YOUR PROGRAM NAME HERE
\newcommand{\authname}{Team \#, Team Name
\\ Student 1 name
\\ Student 2 name
\\ Student 3 name
\\ Student 4 name} % AUTHOR NAMES                  

\usepackage{hyperref}
    \hypersetup{colorlinks=true, linkcolor=blue, citecolor=blue, filecolor=blue,
                urlcolor=blue, unicode=false}
    \urlstyle{same}
                                


\begin{document}

    \maketitle

    \begin{table}[hp]
        \caption{Revision History} \label{TblRevisionHistory}
        \begin{tabularx}{\textwidth}{llX}
            \toprule
            \textbf{Date} & \textbf{Developer(s)} & \textbf{Change}\\
            \midrule
            Date1 & Name(s) & Description of changes\\
            Date2 & Name(s) & Description of changes\\
            ... & ... & ...\\
            \bottomrule
        \end{tabularx}
    \end{table}

    \section{Problem Statement}

    \subsection{Problem}

    \subsection{Inputs and Outputs}

    \subsection{Stakeholders}

    \subsection{Environment}

    \section{Goals}

    \begin{table}[h!]
        \caption{Minimum Viable Product (MVP) Goals} \label{TblMVPGoals}
        \begin{tabularx}{\textwidth}{|l|X|X|}
            \hline
            \textbf{Goal} & \textbf{Explanation} & \textbf{Reasoning} \\
            \hline
            Two-Player Core Loop & Support a one-versus-one match with turn-taking, drawing, discarding, and win condition checks. Includes starter card, discard pile, and reshuffling the stock pile. & Two players represent the smallest playable unit. Completing this ensures the core gameplay is functional and testable. \\
            \hline
            Classic Rules Engine & Implement the standard Crazy Eights rules: match by suit or rank, “8” acts as wild, and drawing occurs if no valid move exists. & Ensuring correctness of the classic game establishes a solid baseline before adding variations. \\
            \hline
            Dozenal (Base-12) Scoring/Display & Display scores, counters, or thresholds in base-12 notation while keeping classic rules unchanged. & Introduces dozenal in a simple, non-disruptive way that highlights novelty while retaining accessibility. \\
            \hline
            Move Validation and Feedback & Provide immediate feedback for invalid moves, suit selection UI after playing an “8,” and clear state indicators. & Reduces errors, lowers learning curve, and improves user experience. \\
            \hline
            Testability and Determinism & Support seeded shuffling and provide basic logs or replays. & Facilitates unit/integration testing and reproducibility during evaluation. \\
            \hline
            Stability and Performance & Ensure responsive UI (<200ms), no crashes, no deadlocks, and correct reshuffling. & Reliability is the baseline for acceptance and live demonstration. \\
            \hline
            Minimal UI & Provide a desktop or web interface showing hand, discard pile, current state, and dozenal score tracker. & Covers essential user interactions while limiting complexity at MVP stage. \\
            \hline
        \end{tabularx}
    \end{table}

    \section{Stretch Goals}

    \begin{table}[h!]
        \caption{Stretch Goals} \label{TblStretchGoals}
        \begin{tabularx}{\textwidth}{|l|X|X|}
            \hline
            \textbf{Goal} & \textbf{Explanation} & \textbf{Reasoning} \\
            \hline
            3–4 Player Matches & Extend gameplay to support three or more players in a single match. & Crazy Eights is often played with more than two people, which increases replayability. \\
            \hline
            Online Multiplayer & Allow players to create/join rooms and synchronize state across network connections. & Brings the game closer to real-world usage and demonstrates system design capability. \\
            \hline
            Advanced Dozenal Variants & Add optional rule packs where cards related to 12 gain special effects or scoring thresholds use base-12 values. & Deepens the dozenal theme while keeping the classic mode intact. \\
            \hline
            Rule Configurator & Provide toggles for house rules such as stacking eights or alternate scoring methods. & Demonstrates variability management and supports experimentation. \\
            \hline
            Tutorial and Visual Guidance & Include first-game tutorial, invalid-move highlights, and play suggestions. & Lowers the learning curve and improves usability. \\
            \hline
            Save/Replay System & Allow saving and replaying completed games. & Supports debugging, user study, and richer documentation. \\
            \hline
            Cross-Platform Packaging & Deploy as a web app or desktop executable. & Lowers barrier for evaluators and external users to try the system. \\
            \hline
        \end{tabularx}
    \end{table}


    \section{Extras}

    \newpage{}

    \section*{Appendix --- Reflection}

    The purpose of reflection questions is to give you a chance to assess your own
learning and that of your group as a whole, and to find ways to improve in the
future. Reflection is an important part of the learning process.  Reflection is
also an essential component of a successful software development process.  

Reflections are most interesting and useful when they're honest, even if the
stories they tell are imperfect. You will be marked based on your depth of
thought and analysis, and not based on the content of the reflections
themselves. Thus, for full marks we encourage you to answer openly and honestly
and to avoid simply writing ``what you think the evaluator wants to hear.''

Please answer the following questions.  Some questions can be answered on the
team level, but where appropriate, each team member should write their own
response:


    \begin{enumerate}
        \item What went well while writing this deliverable?
        \item What pain points did you experience during this deliverable, and how
        did you resolve them?
        \item How did you and your team adjust the scope of your goals to ensure
        they are suitable for a Capstone project?
    \end{enumerate}

\end{document}
