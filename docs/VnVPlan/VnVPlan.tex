\documentclass[12pt, titlepage]{article}

\usepackage{booktabs}
\usepackage{tabularx}
\usepackage{hyperref}
\usepackage{tabularx}
\usepackage{float}

\hypersetup{
    colorlinks,
    citecolor=blue,
    filecolor=black,
    linkcolor=red,
    urlcolor=blue
}
\usepackage[round]{natbib}

%% Comments

\usepackage{color}

\newif\ifcomments\commentstrue %displays comments
%\newif\ifcomments\commentsfalse %so that comments do not display

\ifcomments
\newcommand{\authornote}[3]{\textcolor{#1}{[#3 ---#2]}}
\newcommand{\todo}[1]{\textcolor{red}{[TODO: #1]}}
\else
\newcommand{\authornote}[3]{}
\newcommand{\todo}[1]{}
\fi

\newcommand{\wss}[1]{\authornote{magenta}{SS}{#1}} 
\newcommand{\plt}[1]{\authornote{cyan}{TPLT}{#1}} %For explanation of the template
\newcommand{\an}[1]{\authornote{cyan}{Author}{#1}}

%% Common Parts

\newcommand{\progname}{ProgName} % PUT YOUR PROGRAM NAME HERE
\newcommand{\authname}{Team \#, Team Name
\\ Student 1 name
\\ Student 2 name
\\ Student 3 name
\\ Student 4 name} % AUTHOR NAMES                  

\usepackage{hyperref}
    \hypersetup{colorlinks=true, linkcolor=blue, citecolor=blue, filecolor=blue,
                urlcolor=blue, unicode=false}
    \urlstyle{same}
                                


\begin{document}

\title{System Verification and Validation Plan for The Crazy Tens} 
\author{
    Team \#25, The Crazy Four \\[1ex]
    Ruida Chen \\
    Ammar Sharbat \\
    Alvin Qian \\
    Jiaming Li
}
\date{\today}
	
\maketitle

\pagenumbering{roman}

\section*{Revision History}

\begin{tabularx}{\textwidth}{p{3cm}p{2cm}X}
\toprule {\bf Date} & {\bf Version} & {\bf Notes}\\
\midrule
Oct 24 & Alvin & System test functional requirements\\
Oct 26 & Ruida & Plan\\
Oct 27 & Jiaming & System test Non-functional requirements\\
Oct 27 & Alvin & Fixed errors in FR tests\\
Oct 27 & Ruida & Fixed errors in Plan \\
\bottomrule
\end{tabularx}


\newpage

\tableofcontents

\listoftables
\wss{Remove this section if it isn't needed}

\listoffigures
\wss{Remove this section if it isn't needed}

\newpage

\section{Symbols, Abbreviations, and Acronyms}

\renewcommand{\arraystretch}{1.2}
\begin{tabular}{l l} 
  \toprule		
  \textbf{symbol} & \textbf{description}\\
  \midrule 
  T & Test\\
  \bottomrule
\end{tabular}\\

\wss{symbols, abbreviations, or acronyms --- you can simply reference the SRS
  \citep{SRS} tables, if appropriate}

\wss{Remove this section if it isn't needed}

\newpage

\pagenumbering{arabic}

This document ... \wss{provide an introductory blurb and roadmap of the
  Verification and Validation plan}

\section{General Information}

\subsection{Summary}

\wss{Say what software is being tested.  Give its name and a brief overview of
  its general functions.}

\subsection{Objectives}

\wss{State what is intended to be accomplished.  The objective will be around
  the qualities that are most important for your project.  You might have
  something like: ``build confidence in the software correctness,''
  ``demonstrate adequate usability.'' etc.  You won't list all of the qualities,
  just those that are most important.}

\wss{You should also list the objectives that are out of scope.  You don't have 
the resources to do everything, so what will you be leaving out.  For instance, 
if you are not going to verify the quality of usability, state this.  It is also 
worthwhile to justify why the objectives are left out.}

\wss{The objectives are important because they highlight that you are aware of 
limitations in your resources for verification and validation.  You can't do everything, 
so what are you going to prioritize?  As an example, if your system depends on an 
external library, you can explicitly state that you will assume that external library 
has already been verified by its implementation team.}

\subsection{Challenge Level and Extras}

\wss{State the challenge level (advanced, general, basic) for your project.
Your challenge level should exactly match what is included in your problem
statement.  This should be the challenge level agreed on between you and the
course instructor.  You can use a pull request to update your challenge level
(in TeamComposition.csv or Repos.csv) if your plan changes as a result of the
VnV planning exercise.}

\wss{Summarize the extras (if any) that were tackled by this project.  Extras
can include usability testing, code walkthroughs, user documentation, formal
proof, GenderMag personas, Design Thinking, etc.  Extras should have already
been approved by the course instructor as included in your problem statement.
You can use a pull request to update your extras (in TeamComposition.csv or
Repos.csv) if your plan changes as a result of the VnV planning exercise.}

\subsection{Relevant Documentation}

\wss{Reference relevant documentation.  This will definitely include your SRS
  and your other project documents (design documents, like MG, MIS, etc).  You
  can include these even before they are written, since by the time the project
  is done, they will be written.  You can create BibTeX entries for your
  documents and within those entries include a hyperlink to the documents.}

\citet{SRS}

\wss{Don't just list the other documents.  You should explain why they are relevant and 
how they relate to your VnV efforts.}

\section{Plan}
This section details the comprehensive, multi-phase plan for the verification and validation (V\&V) of the " The Crazy Tens" educational card game, an application designed to teach different number bases through a variant of the Crazy Tens card game. It establishes the V\&V-related responsibilities of the development team, the processes for verifying all key project artifacts (Software Requirements Specification, System Design, Implementation), the specific tools to be leveraged for automation, and the validation strategy to ensure the final product meets its core educational and functional objectives.   



\subsection{Verification and Validation Team}

\begin{table}[H]
\centering
\caption{Verification and Validation Team Roles and Responsibilities}
\label{tab:vvteam}
\begin{tabularx}{\textwidth}{lX}
\toprule
\textbf{Name} & \textbf{Responsibilities} \\
\midrule
Ruida Chen & Owns SRS checklist and requirement traceability; leads ambiguity inspections; opens and triages SRS issues on GitHub. \\
Ammar Sharbat & Conducts MG/MIS walkthroughs; checks FR/NFR coverage; verifies pre/post-conditions in design. \\
Alvin Qian & Develops unit and integration tests; maintains CI coverage metrics; triages static analysis warnings. \\
Jiaming Li & Organizes stakeholder playtests and usability sessions; aligns validation with personas and client expectations. \\
\bottomrule
\end{tabularx}
\end{table}

\subsection{SRS Verification}
\textbf{Objective:}  
To ensure the Software Requirements Specification (SRS) is complete, consistent, verifiable, and aligns with the project goals and constraints defined by the client and supervisor.

\noindent \textbf{Scope:}  
This verification applies to all functional and non-functional requirements described in the SRS. It covers requirement clarity, consistency, completeness, and testability.  

\noindent \textbf{Supervisor Involvement:}  
The SRS verification process will be conducted under the supervision of \textbf{Dr.\ Paul Rapoport}.  
Dr.\ Rapoport will act as the primary reviewer and domain expert, ensuring that the documented requirements correctly reflect the intended educational goals and gameplay design.  
He will provide written and verbal feedback during the SRS review meeting and confirm whether the requirements meet the course expectations.  
Any clarifications or corrections raised by Dr.\ Rapoport will be recorded in the SRS issue tracker and addressed.

\noindent \textbf{Verification Techniques:}
\begin{itemize}
    \item \textbf{Checklist-based inspection:} Each requirement will be reviewed using a formal checklist to ensure it is unambiguous, measurable, and testable.  
    \item \textbf{Cross-traceability review:} Every functional requirement (FR) will be linked to at least one design component and one planned test case.  
    \item \textbf{Peer review:} Team members who did not author the section will perform independent reviews to prevent bias.  
    \item \textbf{Supervisor review:} Dr.\ Rapoport will conduct a targeted review session to evaluate correctness, completeness, and educational intent.
\end{itemize}

\noindent\textbf{Supervisor Involvement:}  
The SRS verification process will be formally reviewed under the supervision of \textbf{Dr.\ Paul Rapoport}.  
The review will take the form of a scheduled meeting lasting approximately 30--45 minutes during the week following the internal peer review.  

\begin{itemize}
    \item \textbf{Meeting preparation:}  
    Before the meeting, the team will prepare and send Dr.\ Rapoport a concise review package containing:
    \begin{enumerate}
        \item The finalized SRS document (PDF);  
        \item A one-page summary of critical or ambiguous requirements;  
        \item The SRS verification checklist used in internal inspection;  
        \item The traceability matrix linking requirements to design/test artifacts.
    \end{enumerate}

    \item \textbf{Structured discussion topics:}  
    During the meeting, the team will:
    \begin{enumerate}
        \item Present how each major functional area (Game, Scoring, Educational Support, User Management) meets the educational and usability goals;
        \item Ask Dr.\ Rapoport to identify unclear, incomplete, or pedagogically inconsistent requirements;
        \item Confirm mutual understanding of success criteria and measurable outcomes (e.g., gameplay flow, scoring accuracy, base conversion learning intent).
    \end{enumerate}

    \item \textbf{Feedback capture and follow-up:}  
    Comments and action items from the meeting will be recorded in the GitHub Issue Tracker under the tag \texttt{SRS-Review}, each with an assigned owner and deadline.  
    The SRS Lead (Ruida Chen) will ensure all feedback is incorporated into the updated SRS before sign-off.

    \item \textbf{Supervisor:}  
    After corrections, Dr.\ Rapoport will be provided with a change summary.
\end{itemize}
\noindent \textbf{Checklist}

\begin{table}[H]
\centering
\caption{SRS Verification Checklist}
\label{tab:srs-checklist}
\begin{tabular}{|l|l|p{0.4\textwidth}|l|l|}
\hline
\textbf{Checklist Item ID} & \textbf{Category} & \textbf{Verification Question} \\ \hline
SRS-C-01 & Completeness & Have all Open Issues been resolved and their resolutions integrated?  \\ \hline
SRS-C-02 & Completeness & Has all placeholder content been replaced with substantive text? \\ \hline
SRS-C-03 & Testability & Does every functional requirement have a unique identifier? \\ \hline
SRS-C-04 & Testability & Does every requirement now have a specific, measurable, and unambiguous Fit Criterion? \\ \hline
SRS-C-05 & Consistency & Are the game rules consistent? (e.g., Is FR-3 consistent with PUC-5?)  \\ \hline
SRS-C-06 & Consistency & Is the technology stack (React, Node.js, Postgres) consistent across Sec 3.1, 13.2, and 26.1? \\ \hline
SRS-C-07 & Correctness & Do all requirements trace back to a stated Project Goal or Stakeholder? \\ \hline
SRS-C-08 & Traceability & Is the scope unambiguous? \\ \hline
SRS-C-09 & Feasibility & Can all requirements be realistically implemented within the project constraints? \\ \hline
\end{tabular}
\end{table}


\subsection{Design Verification}
\noindent \textbf{Objective}: To verify that the system design is a correct, complete, and feasible implementation of the verified SRS

\noindent \textbf{Process:} Design Walkthroughs \& Peer Review
\begin{itemize}
	\item The team will create design artifacts such as UML diagram, database Enitity-Relationship-Diagram (ERD), etc.
	\item Schedule formal review meetings
	\item In these meetings, the team will "walk through" the design, explaining how it fullfills specific requirements from the SRS.
	\item The team will use the checklist below to challenge the design and ensure full traceability from requirements to design.
	\item All identified defects will be logged as GitHub Issues.
\end{itemize}

\noindent \textbf{Checklist}
\begin{table}[H]
\centering
\caption{Design Verification Checklist}
\label{tab:design-checklist}
\begin{tabular}{|l|l|p{0.4\textwidth}|l|l|}
\hline
\textbf{Checklist Item ID} & \textbf{Category} & \textbf{Verification Question}  \\ \hline
DS-C-01 & Traceability & Does every component in the architecture map to one or more Functional Requirements? \\ \hline
DS-C-02 & Completeness & Does the design for the "Game Manager" (Sec 9.1) component explicitly implement all rules)? \\ \hline
DS-C-03 & NFR-Performance & Does the architecture support the 50 concurrent user requirement? \\ \hline
DS-C-04 & NFR-Security & Does the design explicitly address all security requirements, including "server-authoritative game state" ?  \\ \hline
DS-C-05 & NFR-Maintain. & Does the design adhere to the mandated stack (React, Node.js)?[1] Is it modular and extensible ?  \\ \hline
DS-C-06 & NFR-Usability & Does the UI/UX design explicitly address Accessibility and Learning?  \\ \hline
DS-C-07 & Interface & Are all API endpoints clearly defined?  \\ \hline
DS-C-08 & Data & Does the database schema (ERD) correctly model the Business Data Model (Sec 7.1) and Data Dictionary (Sec 7.2)? \\ \hline
\end{tabular}
\end{table}

\subsection{Verification and Validation Plan Verification}

\noindent \textbf{Objective}: To verify this VnVPlan for completeness, feasibility, internal consistency, and adherence to all course rubrics.

\noindent \textbf{Process:} Formal Peer Review
\begin{itemize}
	\item The entire teamshall conduct one full read-through of this document before submitting it.
	\item The team will use the checklist below, which is directly derived from the course grading rubric, to ensure all requirements have been met.
	\item After the team review, the plan shall be informally presented to the course supervisor to get a "feasibility check" before implementation begins, mitigating project risk.
\end{itemize}

\noindent \textbf:{Checklist}
\begin{table}[H]
\centering
\caption{V\&V Plan Verification Checklist}
\label{tab:vvp-checklist}
\begin{tabular}{|l|p{0.5\textwidth}|l|l|}
\hline
\textbf{Checklist Item ID} & \textbf{Verification Question} \\ \hline
VVP-C-01 & Are team roles for V\&V clear, specific, and feasible? \\ \hline
VVP-C-02 & Is the SRS verification plan clear, specific, and feasible? \\ \hline
VVP-C-03 & Is the Design verification plan clear, specific, and feasible? \\ \hline
VVP-C-04 & Is the Implementation verification plan clear, specific, and feasible? \\ \hline
VVP-C-05 & Are the automated testing and verification tools specific?  \\ \hline
VVP-C-06 & Is the software validation plan clear, specific, and feasible?  \\ \hline
VVP-C-08 & Is the plan feasible given the team size and academic term timeline?  \\ \hline
\end{tabular}
\end{table}

\subsection{Implementation Verification}

\noindent \textbf{Objective}: To verify that the source code artifacts correctly implement the verified design, adhere to all coding standards, are free of common defects, and meet all non-functional requirements.

\noindent \textbf{Unit testing}
\begin{itemize}
	\item \textbf{Frontend}:All UI components, verifying correct rendering and user-interaction behavior.
	\item \textbf{Backend}: All API endpoints, authentication logic, and data models.
	\item \textbf{Game Logic}: This is the most critical module to unit test. Tests will cover every game rule, include happy path + all edge cases.

\end{itemize}
\noindent \textbf{Integration testing}
\begin{itemize}
	\item \textbf{Frontend}:API + DB (login/session/history), engine + UI interaction (play card → rule check → UI update).
\end{itemize}

\noindent \textbf{Static Verification:}  
    In addition to dynamic testing, static verification activities will include:
    \begin{itemize}
        \item \textbf{Code walk-throughs:} Weekly peer reviews where presents newly developed code to the team, explaining logic and verifying alignment with design contracts.  
        \item \textbf{Code inspection:} Checklist-based inspections for coding-standard compliance (naming, commenting, error handling).  
        \item \textbf{Static analyzers:}  
        Use of TypeScript’s --strict mode, ESLint rules, and CodeQL analysis for detecting unused variables, type mismatches, and security issues.  
        \item \textbf{Continuous Integration enforcement:}  
        The CI pipeline will fail automatically if linting, type checking, or static analysis errors occur.
    \end{itemize}


\subsection{Automated Testing and Verification Tools}

\noindent\textbf{Objective:}  
To ensure that software verification activities are repeatable, efficient, and integrated with the development workflow.

\noindent\textbf{Tools and Automation:}
\begin{itemize}
    \item \textbf{Jest} --- used for unit and integration testing of both frontend and backend components.
    \item \textbf{Playwright} --- performs automated end-to-end testing of the user interface.
    \item \textbf{ESLint \& Prettier} --- automatically enforce coding style and detect syntax or logical issues.
    \item \textbf{GitHub Actions CI/CD} --- runs all tests automatically on each push or pull request; generates coverage and linting reports.
    \item \textbf{Codecov} --- collects and visualizes code coverage results from CI.
\end{itemize}

\noindent\textbf{Automation Outcome:}  
Automated testing ensures that every commit is verified for correctness, code quality, and integration stability before merging into the main branch.

\subsection{Software Validation}

\noindent\textbf{Objective:}  
To confirm that the final system meets user needs, project goals, and the expectations of the client and supervisor.

\noindent\textbf{Scope:}  
Validation focuses on ensuring that the software behaves as intended by users and stakeholders, not only that it was built correctly.

\noindent\textbf{Validation Methods:}
\begin{itemize}
    \item \textbf{Stakeholder Review:}  
    Conduct a demonstration of the working prototype for Dr.\ Paul Rapoport.  
    Feedback will be collected and documented for any changes before final submission.
    
    \item \textbf{Task-based User Testing:}  
    Team members and sample users will perform core gameplay tasks (e.g., joining a lobby, playing a round, viewing results).  
    Observations will be recorded to identify usability or clarity issues.

    \item \textbf{Survey and Feedback:}  
    A short questionnaire will be used to gather feedback on ease of use, performance, and enjoyment.
\end{itemize}

\noindent\textbf{Success Criteria:}
\begin{itemize}
    \item All core gameplay and scoring functions work as described in the SRS.  
    \item Supervisor and users confirm that the software meets educational and functional goals.  
    \item No critical usability or stability issues remain unresolved.
\end{itemize}

\noindent\textbf{Deliverables:}  
Validation summary report including stakeholder feedback, test logs, and any final fixes implemented before release.


\section{System Tests}

This section defines end to end system tests that validate the software against the SRS Functional Requirements (FR-1..FR-17) for the MVP implementation state. Tests are grouped by related functionality to improve traceability and reuse of fixtures. Unless stated otherwise, tests assume a deterministic deck seed, two authenticated test users, and a clean database. All expected results are observable via the UI and server state exposed to the test harness.

\subsection{Tests for Functional Requirements}

The subsections below cover: (1) game flow, rule validation, special cards, and end conditions (SRS Game Manager: FR-1..FR-5), (2) scoring and valid move highlighting (SRS Score Manager and Highlighting: FR-6..FR-9), (3) account access flows (Login Manager: FR-10..FR-13), and (4) persistence operations (Data Manager: FR-14..FR-17). Where tests rely on input constraints, they use the card value and move legality definitions in the SRS.

\subsubsection{Game Manager: Start, Turn, Rules, Specials, End (FR-1..FR-5)}

This area verifies new game initialization, turn sequencing, rule validation for same suit or same value or sum equals 12 (base-12), handling of wild tens, and detecting the end of a game. It directly traces to FR-1 Start new game, FR-2 Turn management, FR-3 Rule validation, FR-4 Special cards, FR-5 End of game.

\paragraph{Start new game initializes state (ST-GM-01)}
\begin{enumerate}
\item{ST-GM-01\\}
\textbf{Control:} Automatic

\textbf{Initial State:} No active sessions. Two users are authenticated and in the lobby. Deterministic deck seed S is configured.

\textbf{Input:} User A clicks New Game and invites User B. User B accepts.

\textbf{Output:} A new session is created with a shuffled deck using seed S, a discard starter card is placed, each player receives the configured hand size H, and the UI shows turn = User A.

\textbf{Test Case Derivation:} From FR-1 and SRS deck initialization semantics. With a fixed seed S, the initial state is uniquely determined.

\textbf{How test will be performed:} Run a Node console harness that calls \texttt{initGame('S')}, then logs and checks via simple equality that \texttt{hands} have size H, \texttt{discard.length == 1}, and \texttt{turn == UserA}. No browser, no sockets.
\end{enumerate}

\paragraph{Legal move: same suit (ST-GM-02)}
\begin{enumerate}
\item{ST-GM-02\\}
\textbf{Control:} Automatic

\textbf{Initial State:} Active session. Discard top = 5$\heartsuit$. User A hand contains K$\heartsuit$ and others. Turn = User A.

\textbf{Input:} User A plays K$\heartsuit$.

\textbf{Output:} Move is accepted. Discard top becomes K$\heartsuit$. Turn advances to User B.

\textbf{Test Case Derivation:} FR-3 allows a move when suits match.

\textbf{How test will be performed:} In the console harness set \texttt{discard = [{suit:'H',v:5}]} and handA contains K$\heartsuit$. Call \texttt{canPlay()} then \texttt{playCard()}. Log PASS if top becomes K$\heartsuit$ and turn flips to UserB.
\end{enumerate}

\paragraph{Legal move: sum equals 12 (base-12) (ST-GM-03)}
\begin{enumerate}
\item{ST-GM-03\\}
\textbf{Control:} Automatic

\textbf{Initial State:} Discard top = 5$\diamondsuit$ (value 5). User A hand contains 7$\spadesuit$ (value 7). Turn = User A.

\textbf{Input:} User A plays 7$\spadesuit$.

\textbf{Output:} Move is accepted. Discard top becomes 7$\spadesuit$. Turn advances to User B.

\textbf{Test Case Derivation:} In base-12, 5 + 7 = 10$_{12}$ which equals 12 in decimal. FR-3 permits play when values sum to 12 base-12.

\textbf{How test will be performed:} Set top to value 5 then attempt to play value 7. Use \texttt{canPlay()} to assert true and log PASS.
\end{enumerate}

\paragraph{Special card: wild eight suit selection (ST-GM-04)}
\begin{enumerate}
\item{ST-GM-04\\}
\textbf{Control:} Automatic

\textbf{Initial State:} Discard top = K$\clubsuit$. User A hand contains 8$\heartsuit$. Turn = User A.

\textbf{Input:} User A plays 8$\heartsuit$ and selects Spades as the new suit.

\textbf{Output:} Move is accepted. Discard shows 8 with chosen suit indicator = Spades. Only Spades or otherwise legal responses are allowed for the next player.

\textbf{Test Case Derivation:} FR-4 specifies 8s are wild and allow the player to declare a suit.

\textbf{How test will be performed:} Give A an 8, play with \linebreak 
\texttt{playCard(state,'A',eight,\{chooseSuit:'S'\})}, then check \texttt{chosenSuit == 'S'} and \texttt{turn == 'B'} via console logs.
\end{enumerate}

\paragraph{Illegal move is rejected with guidance (ST-GM-05)}
\begin{enumerate}
\item{ST-GM-05\\}
\textbf{Control:} Automatic

\textbf{Initial State:} Discard top = 9$\diamondsuit$. User A hand contains Q$\clubsuit$ and no other legal card.

\textbf{Input:} User A attempts to play Q$\clubsuit$.

\textbf{Output:} Move is rejected. UI shows a polite message explaining why the move is invalid and highlights available actions: draw or play a legal card if any.

\textbf{Test Case Derivation:} FR-3 rejects moves that are not same suit, not same value, and do not sum to 12 base-12.

\textbf{How test will be performed:} Call \texttt{canPlay()} on an invalid card and expect \texttt{false}. For now we just assert the engine rejects the move; UI message will be verified later in UI tests.
\end{enumerate}

\paragraph{End of game detection (ST-GM-06)}
\begin{enumerate}
\item{ST-GM-06\\}
\textbf{Control:} Automatic

\textbf{Initial State:} User A has 1 card, which is legal to play. User B has multiple cards.

\textbf{Input:} User A plays the last card.

\textbf{Output:} Game ends. Winner = User A. System transitions to score calculation and displays results.

\textbf{Test Case Derivation:} FR-5 requires ending the game when a player has no cards left.

\textbf{How test will be performed:} Have a scripted state where A has one legal card, call \texttt{playCard()}, then log PASS if \texttt{state.status == 'Completed'} and \texttt{winner == 'A'}.
\end{enumerate}

\subsubsection{Score Features: Dual-base, Highlights (FR-6..FR-9)}

This area validates round scoring in decimal and Dozenal, correct conversion and display, and visual highlighting of valid moves. Education Support and hints are out of scope.

\paragraph{Score calculated and shown in decimal and Dozenal (ST-SC-01)}
\begin{enumerate}
\item{ST-SC-01\\}
\textbf{Control:} Automatic

\textbf{Initial State:} Game just ended. Opponent remaining cards have numeric values 2, 4, 6 (per SRS value function).

\textbf{Input:} Trigger score calculation screen.

\textbf{Output:} Decimal total = 12. Dozenal total = 10$_{12}$. Both are displayed side by side.

\textbf{Test Case Derivation:} 2 + 4 + 6 = 12 decimal which equals 10 in base-12.

\textbf{How test will be performed:} Call \texttt{calculateScore(\{B:[2,4,6]\})} and expect \texttt{\{ decimal:12, dozenal:'10' \}}. Log PASS or FAIL.
\end{enumerate}

\paragraph{Valid moves highlighted exactly (ST-SC-03)}
\begin{enumerate}
\item{ST-SC-03\\}
\textbf{Control:} Automatic

\textbf{Initial State:} Discard top = 4$\spadesuit$. Hand contains 4$\clubsuit$, 8$\diamondsuit$, 7$\heartsuit$, Q$\spadesuit$, 6$\clubsuit$.

\textbf{Input:} None beyond rendering the hand.

\textbf{Output:} Highlighted cards are exactly \{4$\clubsuit$, Q$\spadesuit$, 8$\diamondsuit$\}. 7$\heartsuit$ and 6$\clubsuit$ are not highlighted.

\textbf{Test Case Derivation:} Same value (4), same suit (Spades), and wild eight are valid. Others are invalid because 7 + 4 = 11 decimal which is not 10$_{12}$ and 6 + 4 = 10 decimal which is not 10$_{12}$.

\textbf{How test will be performed:} Call exported \texttt{listValidMoves(state, hand)} and assert it returns exactly the IDs for same value, same suit, and wild eight. Log PASS or FAIL.
\end{enumerate}

\paragraph{Primary base toggle affects labeling only (ST-SC-04)}
\begin{enumerate}
\item{ST-SC-04\\}
\textbf{Control:} Automatic

\textbf{Initial State:} Score view shows decimal on the left and Dozenal on the right with decimal as primary.

\textbf{Input:} Toggle primary base to Dozenal.

\textbf{Output:} Labels and prominence swap, but numeric values remain identical pair \{12, 10$_{12}$\}. No recomputation changes total values.

\textbf{Test Case Derivation:} FR-7 specifies dual display; the toggle is a presentation choice.

\textbf{How test will be performed:} For MVP engine tests, verify the conversion outputs are stable. The UI toggle is deferred; assert that conversion of 12 dec remains $10_{12}$ before and after calling a \texttt{setPrimaryBase()} no-op stub.
\end{enumerate}

\subsubsection{Login Manager: Account, Sessions, Guest, Validation (FR-10..FR-13)}

This area verifies account creation, login and logout, guest sessions, and credential checks.

\paragraph{Account creation creates unique user and session (ST-LM-01)}
\begin{enumerate}
\item{ST-LM-01\\}
\textbf{Control:} Automatic

\textbf{Initial State:} No account exists for username \texttt{U}. Clean auth store.

\textbf{Input:} Submit sign up with \texttt{username=U}, \texttt{password=P} that meets policy.

\textbf{Output:} Account is created with unique \texttt{userId}. Password is stored as a salted hash (not plaintext). A logged-in session for \texttt{U} is active.

\textbf{Test Case Derivation:} FR-10 requires creation with unique username and password.

\textbf{How test will be performed:} In Node harness call \texttt{createAccount(U,P)}. Assert \texttt{store.users[U].passwordHash} exists and \texttt{store.sessions.has(userId)}. Then attempt a second \texttt{createAccount(U,P2)} and assert rejection.
\end{enumerate}

\paragraph{Login and logout preserve progress (ST-LM-02)}
\begin{enumerate}
\item{ST-LM-02\\}
\textbf{Control:} Automatic

\textbf{Initial State:} Account \texttt{U} exists with prior game history \texttt{H0}. No active session.

\textbf{Input:} \texttt{login(U,P)} then start a game, finish one hand, \texttt{logout()}, then \texttt{login(U,P)}.

\textbf{Output:} Session is established on both logins. After re-login, history includes prior \texttt{H0} plus the new hand \texttt{H1}. No progress is lost at logout.

\textbf{Test Case Derivation:} FR-11 requires login and logout at any time without losing progress.

\textbf{How test will be performed:} Call \texttt{login}, run minimal game through engine hooks, call \texttt{logout}, then \texttt{login} again. Compare history count before and after.
\end{enumerate}

\paragraph{Guest mode creates temporary session (ST-LM-03)}
\begin{enumerate}
\item{ST-LM-03\\}
\textbf{Control:} Automatic

\textbf{Initial State:} No active session.

\textbf{Input:} \texttt{createGuestSession()} then play a short hand.

\textbf{Output:} Session has \texttt{guest:true}. No persistent user record is created. On process restart or \texttt{endGuestSession()}, the profile is not retrievable.

\textbf{Test Case Derivation:} FR-12 allows a temporary session without login.

\textbf{How test will be performed:} Call \texttt{createGuestSession()}, verify \texttt{session.guest == true}. Write a hand, restart in-memory store, assert no user profile exists for the guest.
\end{enumerate}

\paragraph{Credential validation gates access (ST-LM-04)}
\begin{enumerate}
\item{ST-LM-04\\}
\textbf{Control:} Automatic

\textbf{Initial State:} Account \texttt{U} exists with password \texttt{P}. No active session.

\textbf{Input:} Attempt \texttt{login(U,'wrong')}, then \texttt{login(U,P)}.

\textbf{Output:} First attempt is rejected with an auth error. No session is created. Second attempt succeeds and creates a session.

\textbf{Test Case Derivation:} FR-13 requires validating credentials against stored records before granting access.

\textbf{How test will be performed:} Call \texttt{login} with wrong password and assert failure and \texttt{sessions.size} unchanged, then with correct password and assert success.
\end{enumerate}

\subsubsection{Data Manager: Storage, Retrieval, Update, Deletion (FR-14..FR-17)}

This area verifies secure storage of usernames, game history, and Dozenal scores, along with retrieval, update, and deletion.

\paragraph{User data is stored on events that require persistence (ST-DM-01)}
\begin{enumerate}
\item{ST-DM-01\\}
\textbf{Control:} Automatic

\textbf{Initial State:} Authenticated user \texttt{U}. Clean data store.

\textbf{Input:} Finish a game where opponent holds cards with values \{2,4,6\}. Trigger save.

\textbf{Output:} A record exists for \texttt{U} containing username, a new game history entry with decimal total 12 and Dozenal 10$_{12}$.

\textbf{Test Case Derivation:} FR-14 requires secure storage of user data including game history and Dozenal scores.

\textbf{How test will be performed:} Call \texttt{saveGameResult(U, result)}. Assert schema fields exist and values match expected totals. For MVP we verify presence and correctness; deeper security assessments are separate.
\end{enumerate}

\paragraph{Stored data can be retrieved on login or profile view (ST-DM-02)}
\begin{enumerate}
\item{ST-DM-02\\}
\textbf{Control:} Automatic

\textbf{Initial State:} Data store contains prior records for \texttt{U}.

\textbf{Input:} \texttt{login(U,P)} then \texttt{getProfile(U)}.

\textbf{Output:} Retrieved profile includes username, cumulative game history, and Dozenal scores as saved.

\textbf{Test Case Derivation:} FR-15 requires retrieval of stored user data.

\textbf{How test will be performed:} Assert \texttt{getProfile(U)} returns non-empty history and last score equals most recent saved value.
\end{enumerate}

\paragraph{Data updates after games and profile changes (ST-DM-03)}
\begin{enumerate}
\item{ST-DM-03\\}
\textbf{Control:} Automatic

\textbf{Initial State:} Profile for \texttt{U} exists with history length \texttt{n}.

\textbf{Input:} Complete another game and call \texttt{saveGameResult}. Then change profile display name via \texttt{updateProfile(U,\{displayName:'A'\})}.

\textbf{Output:} History length becomes \texttt{n+1}. Profile \texttt{displayName} is updated. Timestamps reflect the latest update.

\textbf{Test Case Derivation:} FR-16 requires updating stored data after each game or profile change.

\textbf{How test will be performed:} Compare history length and profile fields before and after the operations.
\end{enumerate}

\paragraph{User can permanently delete data (ST-DM-04)}
\begin{enumerate}
\item{ST-DM-04\\}
\textbf{Control:} Automatic

\textbf{Initial State:} Profile and history exist for \texttt{U}.

\textbf{Input:} \texttt{deleteUserData(U)} then attempt \texttt{getProfile(U)} and \texttt{login(U,P)}.

\textbf{Output:} User records, history, and scores are removed. \texttt{getProfile(U)} returns not found. Login is rejected since the account data has been deleted or deactivated per policy.

\textbf{Test Case Derivation:} FR-17 requires permanent deletion upon request.

\textbf{How test will be performed:} Call \texttt{deleteUserData} and assert data store no longer contains keys for \texttt{U}. Verify subsequent access fails.
\end{enumerate}


\subsection{Tests for Nonfunctional Requirements}

    \subsubsection{Performance and Responsiveness (NFR-1)}
    \textbf{Objective:} Verify that all user interactions, such as playing a card or updating the score, occur with a latency below 200\,ms under normal network conditions. \\

    \textbf{Type:} Dynamic, Automatic \\

    \textbf{Initial State:} Running MVP build on localhost with two authenticated test users and a seeded deck. \\

    \textbf{Input / Condition:} Simulated play sessions of 100 rounds using Playwright scripts. \\

    \textbf{Expected Output / Metric:} Average response time $\leq 200$\,ms, 95th-percentile $\leq 350$\,ms; no dropped WebSocket messages. \\

    \textbf{How test will be performed:}
    Use Playwright performance APIs to measure latency between client events and UI updates. Aggregate results into a summary table through CI (GitHub Actions).
    If thresholds are exceeded, flag the run as a warning and create an issue.

    \subsubsection{Usability and User Satisfaction (NFR-2)}
    \textbf{Objective:} Assess how easily new users can understand and play the Dozenal Crazy Tens game. \\

    \textbf{Type:} Dynamic, Manual \\

    \textbf{Initial State:} Usable prototype deployed on a test server. Five participants who have never played the Dozenal version will take part in a guided session. \\

    \textbf{Input / Condition:} Each participant completes a 15-minute task sequence (start a game, play 3 rounds, view score explanation). \\

    \textbf{Expected Output / Metric:} At least 80\% of participants successfully finish the task without assistance; post-test survey average rating $\geq 4$ (out of 5) on clarity and enjoyment. \\

    \textbf{How test will be performed:}
    Observation and screen recording during gameplay, followed by a 5-question Likert survey (see Appendix 6.2).
    Feedback will be compiled into a usability report.

    \subsubsection{Stability and Robustness (NFR-3)}
    \textbf{Objective:} Confirm that the application handles unexpected network events and invalid inputs gracefully. \\

    \textbf{Type:} Dynamic, Automatic + Manual Fault Injection \\

    \textbf{Initial State:} Running multi-client session connected via WebSocket. \\

    \textbf{Input / Condition:} Simulate network interruptions and send malformed game actions through a test harness. \\

    \textbf{Expected Output / Metric:} Application recovers without crash or data loss; users receive an error message ``Connection lost – reconnecting…'' and game state resyncs within 5 seconds. \\

    \textbf{How test will be performed:}
    Inject disconnects using Playwright’s network emulation tools. Review browser console and server logs for exceptions. Report mean recovery time and failure rate.

    \subsubsection{Maintainability and Code Quality (NFR-4)}
    \textbf{Objective:} Evaluate source-code consistency and test coverage as indicators of maintainability. \\

    \textbf{Type:} Static Analysis / Automated \\

    \textbf{Initial State:} Repository after successful build on GitHub Actions. \\

    \textbf{Input / Condition:} Run ESLint, Prettier, and CodeQL scans; collect coverage data via Codecov. \\

    \textbf{Expected Output / Metric:} No ESLint errors or critical CodeQL alerts; line coverage $\geq 80$\%. \\

    \textbf{How test will be performed:}
    Continuous Integration pipeline executes static analysis and unit testing jobs on each commit. Results are stored as badges in the repository README for transparency.



\subsection{Traceability Between Test Cases and Requirements}

\begin{table}[H]
    \centering
    \caption{Traceability Between Test Cases and Requirements}
    \label{TblTraceability}
    \begin{tabular}{|p{0.2\textwidth}|p{0.45\textwidth}|p{0.35\textwidth}|}
        \hline
        \textbf{Requirement ID} & \textbf{Description (from SRS)} & \textbf{Associated Test Case(s)} \\ \hline

        FR-1 & Start a new game session with two players. & ST-GM-01 \\ \hline
        FR-2 & Manage player turns and enforce alternating actions. & ST-GM-02, ST-GM-04 \\ \hline
        FR-3 & Validate move legality (same suit, same value, or sum equals 12 in base-12). & ST-GM-02, ST-GM-03, ST-GM-05 \\ \hline
        FR-4 & Handle special cards (wild eight suit declaration). & ST-GM-04 \\ \hline
        FR-5 & Detect end-of-game condition and declare winner. & ST-GM-06 \\ \hline
        FR-6 & Compute total score in both decimal and dozenal formats. & ST-SC-01 \\ \hline
        FR-7 & Display score in dual-base format with toggling support. & ST-SC-04 \\ \hline
        FR-8 & Provide Dozenal arithmetic explanations and hints. & ST-SC-02 \\ \hline
        FR-9 & Highlight valid playable cards in the UI. & ST-SC-03 \\ \hline

        NFR-1 & Performance: all user interactions respond in under 200\,ms. & NFR-1 Performance and Responsiveness test \\ \hline
        NFR-2 & Usability: at least 80\% of users complete tasks successfully; survey rating $\geq$ 4/5. & NFR-2 Usability and User Satisfaction test \\ \hline
        NFR-3 & Stability: recovery from disconnects without data loss within 5\,s. & NFR-3 Stability and Robustness test \\ \hline
        NFR-4 & Maintainability: code quality verified via ESLint, CodeQL, and $\geq$80\% test coverage. & NFR-4 Maintainability and Code Quality test \\ \hline

    \end{tabular}
\end{table}

\section{Unit Test Description}

\wss{This section should not be filled in until after the MIS (detailed design
  document) has been completed.}

\wss{Reference your MIS (detailed design document) and explain your overall
philosophy for test case selection.}  

\wss{To save space and time, it may be an option to provide less detail in this section.  
For the unit tests you can potentially layout your testing strategy here.  That is, you 
can explain how tests will be selected for each module.  For instance, your test building 
approach could be test cases for each access program, including one test for normal behaviour 
and as many tests as needed for edge cases.  Rather than create the details of the input 
and output here, you could point to the unit testing code.  For this to work, you code 
needs to be well-documented, with meaningful names for all of the tests.}

\subsection{Unit Testing Scope}

\wss{What modules are outside of the scope.  If there are modules that are
  developed by someone else, then you would say here if you aren't planning on
  verifying them.  There may also be modules that are part of your software, but
  have a lower priority for verification than others.  If this is the case,
  explain your rationale for the ranking of module importance.}

\subsection{Tests for Functional Requirements}

\wss{Most of the verification will be through automated unit testing.  If
  appropriate specific modules can be verified by a non-testing based
  technique.  That can also be documented in this section.}

\subsubsection{Module 1}

\wss{Include a blurb here to explain why the subsections below cover the module.
  References to the MIS would be good.  You will want tests from a black box
  perspective and from a white box perspective.  Explain to the reader how the
  tests were selected.}

\begin{enumerate}

\item{test-id1\\}

Type: \wss{Functional, Dynamic, Manual, Automatic, Static etc. Most will
  be automatic}
					
Initial State: 
					
Input: 
					
Output: \wss{The expected result for the given inputs}

Test Case Derivation: \wss{Justify the expected value given in the Output field}

How test will be performed: 
					
\item{test-id2\\}

Type: \wss{Functional, Dynamic, Manual, Automatic, Static etc. Most will
  be automatic}
					
Initial State: 
					
Input: 
					
Output: \wss{The expected result for the given inputs}

Test Case Derivation: \wss{Justify the expected value given in the Output field}

How test will be performed: 

\item{...\\}
    
\end{enumerate}

\subsubsection{Module 2}

...

\subsection{Tests for Nonfunctional Requirements}

\wss{If there is a module that needs to be independently assessed for
  performance, those test cases can go here.  In some projects, planning for
  nonfunctional tests of units will not be that relevant.}

\wss{These tests may involve collecting performance data from previously
  mentioned functional tests.}

\subsubsection{Module ?}
		
\begin{enumerate}

\item{test-id1\\}

Type: \wss{Functional, Dynamic, Manual, Automatic, Static etc. Most will
  be automatic}
					
Initial State: 
					
Input/Condition: 
					
Output/Result: 
					
How test will be performed: 
					
\item{test-id2\\}

Type: Functional, Dynamic, Manual, Static etc.
					
Initial State: 
					
Input: 
					
Output: 
					
How test will be performed: 

\end{enumerate}

\subsubsection{Module ?}

...

\subsection{Traceability Between Test Cases and Modules}

\wss{Provide evidence that all of the modules have been considered.}
				
\bibliographystyle{plainnat}

\bibliography{../../refs/References}

\newpage

\section{Appendix}

This is where you can place additional information.

\subsection{Symbolic Parameters}

The definition of the test cases will call for SYMBOLIC\_CONSTANTS.
Their values are defined in this section for easy maintenance.

\subsection{Usability Survey Questions?}

\wss{This is a section that would be appropriate for some projects.}

\newpage{}
\section*{Appendix --- Reflection}

\wss{This section is not required for CAS 741}

The information in this section will be used to evaluate the team members on the
graduate attribute of Lifelong Learning.

The purpose of reflection questions is to give you a chance to assess your own
learning and that of your group as a whole, and to find ways to improve in the
future. Reflection is an important part of the learning process.  Reflection is
also an essential component of a successful software development process.  

Reflections are most interesting and useful when they're honest, even if the
stories they tell are imperfect. You will be marked based on your depth of
thought and analysis, and not based on the content of the reflections
themselves. Thus, for full marks we encourage you to answer openly and honestly
and to avoid simply writing ``what you think the evaluator wants to hear.''

Please answer the following questions.  Some questions can be answered on the
team level, but where appropriate, each team member should write their own
response:


\begin{enumerate}
  \item What went well while writing this deliverable? 

During the writing of this deliverable, our team improved our understanding of how verification and validation activities fit into a software lifecycle.  We learned to trace requirements back to specific tests, and maintain consistency across project documents(SRS and Supervisor feedbacks). Overall, this deliverable helped us formalize our testing strategy and clarify the expectations for the Crazy Tens project.
  \item What pain points did you experience during this deliverable, and how
    did you resolve them?


Scope drift around base rules and wild card, cross-doc sync issues between SRS and VnV, and clarifying auth/data flows; we resolved these by parameterizing rules via config and FR ids, adding a small CSV trace matrix and change logs, and specifying salted hashing, session behavior, and delete flows mirrored by tests.
  \item What knowledge and skills will the team collectively need to acquire to
  successfully complete the verification and validation of your project?
  Examples of possible knowledge and skills include dynamic testing knowledge,
  static testing knowledge, specific tool usage, Valgrind etc.  You should look to
  identify at least one item for each team member.

Collectively we need stronger dynamic and static testing and specific tool skills; Alvin: CI and property-based testing coverage, Ruida: requirements-to-tests traceability and mutation testing, Ammar: authentication and data lifecycle verification, Jiaming: UI highlighting and scoring verification with basic accessibility.
  \item For each of the knowledge areas and skills identified in the previous
  question, what are at least two approaches to acquiring the knowledge or
  mastering the skill?  Of the identified approaches, which will each team
  member pursue, and why did they make this choice?

Alvin will set up GitHub Actions and add fast-check property tests for edge cases because it raises coverage with low overhead; Ruida will build a CSV trace matrix; Ammar will implement salted-hash, session, and CRUD delete tests because auth and data are MVP-critical; Jiaming will create a small Playwright suite and add axe checks in one target browser because it keeps scope focused while covering UI correctness and accessibility.
\end{enumerate}

\end{document}