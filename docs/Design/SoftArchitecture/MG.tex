\documentclass[12pt, titlepage]{article}

\usepackage{fullpage}
\usepackage[round]{natbib}
\usepackage{multirow}
\usepackage{booktabs}
\usepackage{tabularx}
\usepackage{graphicx}
\usepackage{float}
\usepackage{hyperref}
\hypersetup{
    colorlinks,
    citecolor=blue,
    filecolor=black,
    linkcolor=red,
    urlcolor=blue
}

\input{../../Comments}
\input{../../Common}

\newcounter{acnum}
\newcommand{\actheacnum}{AC\theacnum}
\newcommand{\acref}[1]{AC\ref{#1}}

\newcounter{ucnum}
\newcommand{\uctheucnum}{UC\theucnum}
\newcommand{\uref}[1]{UC\ref{#1}}

\newcounter{mnum}
\newcommand{\mthemnum}{M\themnum}
\newcommand{\mref}[1]{M\ref{#1}}

\begin{document}

\title{Module Guide for The Crazy Four} 
\author{
    Team \#25, The Crazy Four \\[1ex]
    Ruida Chen \\
    Ammar Sharbat \\
    Alvin Qian \\
    Jiaming Li
}
\date{\today}

\maketitle

\pagenumbering{roman}

\section{Revision History}

\begin{tabularx}{\textwidth}{p{3cm}p{2cm}X}
\toprule {\bf Date} & {\bf Version} & {\bf Notes}\\
\midrule
Nov 8 & Alvin & Module Hierarchy and decomposition\\
Nov 9 & Alvin & Updated API's for game action modules\\

\bottomrule
\end{tabularx}

\newpage

\section{Reference Material}

This section records information for easy reference.

\subsection{Abbreviations and Acronyms}

\renewcommand{\arraystretch}{1.2}
\begin{tabular}{l l} 
  \toprule		
  \textbf{symbol} & \textbf{description}\\
  \midrule 
  AC & Anticipated Change\\
  ACID & Atomicity, Consistency, Isolation, Durability\\
  API & Application Programming Interface\\
  CSS & Cascading Style Sheets\\
  DAG & Directed Acyclic Graph \\
  DOM & Document Object Model\\
  FR & Functional Requirement\\
  HTTP & Hypertext Transfer Protocol\\
  JSON & JavaScript Object Notation\\
  JWT & JSON Web Token\\
  M & Module \\
  MG & Module Guide \\
  NFR & Non-Functional Requirement\\
  OS & Operating System \\
  R & Requirement\\
  REST & Representational State Transfer\\
  SC & Scientific Computing \\
  SR & Safety Requirement\\
  SRS & Software Requirements Specification\\
  SQL & Structured Query Language\\
  UC & Unlikely Change \\
  UI & User Interface\\
  \bottomrule
\end{tabular}\\

\newpage

\tableofcontents

\listoftables

\listoffigures

\newpage

\pagenumbering{arabic}

\section{Introduction}

Decomposing a system into modules is a commonly accepted approach to developing
software.  A module is a work assignment for a programmer or programming
team~\citep{ParnasEtAl1984}.  We advocate a decomposition
based on the principle of information hiding~\citep{Parnas1972a}.  This
principle supports design for change, because the ``secrets'' that each module
hides represent likely future changes.  Design for change is valuable in SC,
where modifications are frequent, especially during initial development as the
solution space is explored.  

Our design follows the rules layed out by \citet{ParnasEtAl1984}, as follows:
\begin{itemize}
\item System details that are likely to change independently should be the
  secrets of separate modules.
\item Each data structure is implemented in only one module.
\item Any other program that requires information stored in a module's data
  structures must obtain it by calling access programs belonging to that module.
\end{itemize}

After completing the first stage of the design, the Software Requirements
Specification (SRS), the Module Guide (MG) is developed~\citep{ParnasEtAl1984}. The MG
specifies the modular structure of the system and is intended to allow both
designers and maintainers to easily identify the parts of the software.  The
potential readers of this document are as follows:

\begin{itemize}
\item New project members: This document can be a guide for a new project member
  to easily understand the overall structure and quickly find the
  relevant modules they are searching for.
\item Maintainers: The hierarchical structure of the module guide improves the
  maintainers' understanding when they need to make changes to the system. It is
  important for a maintainer to update the relevant sections of the document
  after changes have been made.
\item Designers: Once the module guide has been written, it can be used to
  check for consistency, feasibility, and flexibility. Designers can verify the
  system in various ways, such as consistency among modules, feasibility of the
  decomposition, and flexibility of the design.
\end{itemize}

The rest of the document is organized as follows. Section
\ref{SecChange} lists the anticipated and unlikely changes of the software
requirements. Section \ref{SecMH} summarizes the module decomposition that
was constructed according to the likely changes. Section \ref{SecConnection}
specifies the connections between the software requirements and the
modules. Section \ref{SecMD} gives a detailed description of the
modules. Section \ref{SecTM} includes two traceability matrices. One checks
the completeness of the design against the requirements provided in the SRS. The
other shows the relation between anticipated changes and the modules. Section
\ref{SecUse} describes the use relation between modules.

\section{Anticipated and Unlikely Changes} \label{SecChange}

This section lists possible changes to the system. According to the likeliness
of the change, the possible changes are classified into two
categories. Anticipated changes are listed in Section \ref{SecAchange}, and
unlikely changes are listed in Section \ref{SecUchange}.

\subsection{Anticipated Changes} \label{SecAchange}

Anticipated changes are the source of the information that is to be hidden
inside the modules. Ideally, changing one of the anticipated changes will only
require changing the one module that hides the associated decision. The approach
adapted here is called design for
change.

\begin{description}
\item[\refstepcounter{acnum} \actheacnum \label{acHardware}:] The specific
  hardware on which the software is running.
\item[\refstepcounter{acnum} \actheacnum \label{acInput}:] The format of the
  initial input data.
\item ...
\end{description}

\wss{Anticipated changes relate to changes that would be made in requirements,
design or implementation choices.  They are not related to changes that are made
at run-time, like the values of parameters.}

\subsection{Unlikely Changes} \label{SecUchange}

The module design should be as general as possible. However, a general system is
more complex. Sometimes this complexity is not necessary. Fixing some design
decisions at the system architecture stage can simplify the software design. If
these decision should later need to be changed, then many parts of the design
will potentially need to be modified. Hence, it is not intended that these
decisions will be changed.

\begin{description}
\item[\refstepcounter{ucnum} \uctheucnum \label{ucIO}:] Input/Output devices
  (Input: File and/or Keyboard, Output: File, Memory, and/or Screen).
\item ...
\end{description}

\section{Module Hierarchy} \label{SecMH}

This section provides an overview of the module design. Modules are summarized
in a hierarchy decomposed by secrets in Table \ref{TblMH}. The modules listed
below, which are leaves in the hierarchy tree, are the modules that will
actually be implemented.


\begin{description}
% --- Backend Modules ---
\item [\refstepcounter{mnum} \mthemnum \label{mAPI}:] API Module
    \begin{itemize}
        \item Provides stateless HTTP (REST) endpoints for auth and profile management.
    \end{itemize}

\item [\refstepcounter{mnum} \mthemnum \label{mRealtimeGateway}:] Real-time Gateway Module
    \begin{itemize}
        \item Manages stateful WebSocket connections for live gameplay and state syncing.
    \end{itemize}

\item [\refstepcounter{mnum} \mthemnum \label{mMatchmaking}:] Matchmaking Module
    \begin{itemize}
        \item Handles game lobby creation, joining, and starting a match.
    \end{itemize}

\item [\refstepcounter{mnum} \mthemnum \label{mAuth}:] Authentication Module
    \begin{itemize}
        \item Manages user identity, password hashing, and session token generation.
    \end{itemize}

\item [\refstepcounter{mnum} \mthemnum \label{mRepository}:] Repository Module
    \begin{itemize}
        \item Abstracts all database queries (SQL) for creating, reading, updating, and deleting data.
    \end{itemize}

\item [\refstepcounter{mnum} \mthemnum \label{mAudit}:] Audit Module
    \begin{itemize}
        \item Logs important server-side events for debugging and security.
    \end{itemize}

% --- Frontend Modules ---
\item [\refstepcounter{mnum} \mthemnum \label{mRealtimeClient}:] Real-time Client Module
    \begin{itemize}
        \item Establishes and maintains the client-side WebSocket connection; sends/receives game events.
    \end{itemize}

\item [\refstepcounter{mnum} \mthemnum \label{mAppShell}:] Application Shell Module
    \begin{itemize}
        \item The main React component providing global layout, navigation, and state.
    \end{itemize}

\item [\refstepcounter{mnum} \mthemnum \label{mAuthClient}:] Authentication Client Module
    \begin{itemize}
        \item Provides the UI and logic for login/signup forms.
    \end{itemize}

\item [\refstepcounter{mnum} \mthemnum \label{mLobbyView}:] Lobby View Module
    \begin{itemize}
        \item UI component for displaying, creating, and joining game lobbies.
    \end{itemize}

\item [\refstepcounter{mnum} \mthemnum \label{mGameBoardView}:] Game Board View Module
    \begin{itemize}
        \item UI component that renders the main game interface (hands, deck, discard pile).
    \end{itemize}

\item [\refstepcounter{mnum} \mthemnum \label{mMoveController}:] Move Controller Module
    \begin{itemize}
        \item Manages user input (e.g., card clicks) and highlights valid moves.
    \end{itemize}

\item [\refstepcounter{mnum} \mthemnum \label{mScoreboardView}:] Scoreboard View Module
    \begin{itemize}
        \item UI component for displaying end-of-round scores in decimal and Dozenal.
    \end{itemize}

\item [\refstepcounter{mnum} \mthemnum \label{mProfileView}:] Profile View Module
    \begin{itemize}
        \item UI component for displaying user statistics and game history.
    \end{itemize}

% --- Core/Behaviour Modules ---
\item [\refstepcounter{mnum} \mthemnum \label{mGameEngine}:] Game Engine Module
    \begin{itemize}
        \item Manages the core game state (deck, hands) and turn progression.
    \end{itemize}

\item [\refstepcounter{mnum} \mthemnum \label{mRules}:] Rules Module
    \begin{itemize}
        \item Stateless logic to validate moves (e.g., match suit, rank, or Dozenal sum).
    \end{itemize}

\item [\refstepcounter{mnum} \mthemnum \label{mScoring}:] Scoring Module
    \begin{itemize}
        \item Calculates scores at the end of a round.
    \end{itemize}

\item [\refstepcounter{mnum} \mthemnum \label{mBaseConversion}:] Base Conversion Module
    \begin{itemize}
        \item Utility to convert numbers between decimal and Dozenal.
    \end{itemize}

\item [\refstepcounter{mnum} \mthemnum \label{mGameActions}:] Game Actions Module
    \begin{itemize}
        \item Defines types and structure for player actions (play card, draw, declare suit, submit score tally).
    \end{itemize}

% --- Hardware Hiding ---
\item [\refstepcounter{mnum} \mthemnum \label{mOS}:] Operating System Module
    \begin{itemize}
        \item Represents the server's OS, providing the Node.js runtime environment.
    \end{itemize}

\item [\refstepcounter{mnum} \mthemnum \label{mBrowser}:] Browser Runtime Module
    \begin{itemize}
        \item Represents the client's web browser, providing the React runtime environment.
    \end{itemize}

\item [\refstepcounter{mnum} \mthemnum \label{mDB}:] Database Module
    \begin{itemize}
        \item Represents the PostgreSQL software that handles physical data storage.
    \end{itemize}
\end{description}

\begin{table}[H]
\centering
\begin{tabular}{p{0.3\textwidth} p{0.3\textwidth} p{0.3\textwidth}}
\toprule
\textbf{Level 1} & \textbf{Level 2} & \textbf{Level 3 (Leaf Modules)}\\
\midrule

\multirow{3}{0.3\textwidth}{Hardware-Hiding Module} & ~ & \mref{mOS} (Server OS) \\
& ~ & \mref{mBrowser} (Client Runtime) \\
& ~ & \mref{mDB} (PostgreSQL) \\
\midrule

\multirow{5}{0.3\textwidth}{Behaviour-Hiding Module} & (Core Domain Logic) & \mref{mGameEngine} \\
& & \mref{mRules} \\
& & \mref{mScoring} \\
& & \mref{mBaseConversion} \\
& & \mref{mGameActions} \\
\midrule

\multirow{15}{0.3\textwidth}{Software Decision Module} & \multirow{6}{0.3\textwidth}{Backend (Server)} & \mref{mAPI} \\
& & \mref{mRealtimeGateway} \\
& & \mref{mMatchmaking} \\
& & \mref{mAuth} \\
& & \mref{mRepository} \\
& & \mref{mAudit} \\
\cmidrule{2-3}
& \multirow{9}{0.3\textwidth}{Frontend (Client)} & \mref{mRealtimeClient} \\
& & \mref{mAppShell} \\
& & \mref{mAuthClient} \\
& & \mref{mLobbyView} \\
& & \mref{mGameBoardView} \\
& & \mref{mMoveController} \\
& & \mref{mScoreboardView} \\
& & \mref{mProfileView} \\
\bottomrule

\end{tabular}
\caption{Module Hierarchy}
\label{TblMH}
\end{table}

\section{Connection Between Requirements and Design} \label{SecConnection}

The design of the system is intended to satisfy the requirements developed in
the SRS. In this stage, the system is decomposed into modules. The connection
between requirements and modules is listed in Table~\ref{TblRT}.\\


The design of the system is intended to satisfy the requirements developed in
the SRS. In this stage, the system is decomposed into modules. The connection
between requirements and modules is listed in the Traceability Matrix in Section \ref{SecTM} (Table \ref{TblRT}). This decomposition ensures that each Functional Requirement (FR), Non-functional Requirement (NFR), and Safety Requirement (SR) has a clear owner in the design, facilitating implementation and verification.

For example, core gameplay logic (FR-1 to FR-5) is satisfied by the \mref{mGameEngine} and \mref{mRules} modules, while the user-facing presentation (FR-7, FR-9) is handled by frontend modules like \mref{mScoreboardView} and \mref{mGameBoardView}. Security and data persistence requirements (FR-10 to FR-17, SR-3, SR-8) are satisfied by the backend's \mref{mAuth} and \mref{mRepository} modules.

\section{Module Decomposition} \label{SecMD}

Modules are decomposed according to the principle of ``information hiding''
proposed by \citet{ParnasEtAl1984}. The \emph{Secrets} field in a module
decomposition is a brief statement of the design decision hidden by the
module. The \emph{Services} field specifies \emph{what} the module will do
without documenting \emph{how} to do it. For each module, a suggestion for the
implementing software is given under the \emph{Implemented By} title.

Only the leaf modules in the hierarchy have to be implemented.

\subsection{Hardware Hiding Modules}

\subsubsection{Operating System Module (\mref{mOS})}
\begin{quote}
\begin{description}
  \item[Secrets:] Process scheduling, filesystem, Node.js runtime environment, network stack.
  \item[Services:] Provides the execution environment for the backend server.
  \item[Implemented By:] OS (such as Linux)
\end{description}
\end{quote}

\subsubsection{Browser Runtime Module (\mref{mBrowser})}
\begin{quote}
\begin{description}
  \item[Secrets:] DOM rendering, event loop, TypeScript (React) execution, WebSocket/HTTP client implementation.
  \item[Services:] Provides the execution environment for the frontend client.
  \item[Implemented By:] Browser (Chrome, Firefox, Edge)
\end{description}
\end{quote}

\subsubsection{Database Module (\mref{mDB})}
\begin{quote}
\begin{description}
  \item[Secrets:] Data storage on disk, indexing, transaction (ACID) implementation, SQL query optimization.
  \item[Services:] Provides persistent storage for user and game data.
  \item[Implemented By:] PostgreSQL
\end{description}
\end{quote}

\subsection{Behaviour-Hiding Module}

\subsubsection{Game Engine Module (\mref{mGameEngine})}
\begin{quote}
\begin{description}
  \item[Secrets:] The data structure representing the game state (players, deck, discard pile, turn). The state machine for turn management.
  \item[Services:] Initializes a new game, applies a validated move to the game state, manages turn progression, detects end-of-game conditions.
  \item[Implemented By:] The Crazy Four (TypeScript)
  \item[State Variables:] `gameState`: \{ `players`: [ ], `deck`: Card[ ], `discardPile`: Card[ ], `currentTurn`: PlayerID, `status`: string \}
  \item[Access Programs:] `dealOpeningHands(players, options)` $\to$ `GameState`; `playCard(gameState, playerID, cardID)` $\to$ `GameState`; `drawCard(gameState, playerID)` $\to$ `GameState`; `applyDeclaredSuit(gameState, playerID, suit)` $\to$ `GameState`; `finalizeRound(gameState)` $\to$ `ScoreSummary`
  \item[Exceptions:] `InvalidGameStateError`
\end{description}
\end{quote}

\subsubsection{Rules Module (\mref{mRules})}
\begin{quote}
\begin{description}
  \item[Secrets:] The specific logic for move validation: same suit, same rank, or sum equals 12 (base-12). The logic for special cards (e.g., '10' is wild).
  \item[Services:] Validates if a move is legal given the current game state. Lists all legal moves for a player.
  \item[Implemented By:] The Crazy Four (TypeScript)
  \item[State Variables:] None (pure functions).
  \item[Access Programs:] `isLegalMove(move, gameState)` $\to$ `bool`; `getValidMoves(playerHand, gameState)` $\to$ `Card[]`
  \item[Exceptions:] `InvalidRuleConfig`
\end{description}
\end{quote}

\subsubsection{Scoring Module (\mref{mScoring})}
\begin{quote}
\begin{description}
  \item[Secrets:] The algorithm for calculating round scores based on opponents' remaining cards.
  \item[Services:] Calculates the score for a completed round.
  \item[Implemented By:] The Crazy Four (TypeScript)
  \item[State Variables:] None (pure functions).
  \item[Access Programs:] `calculateRoundScore(winningPlayerID, gameState)` $\to$ `ScoreObject`
  \item[Exceptions:] None
\end{description}
\end{quote}

\subsubsection{Base Conversion Module (\mref{mBaseConversion})}
\begin{quote}
\begin{description}
  \item[Secrets:] The algorithm and symbols (e.g., \reflectbox{2}, \reflectbox{3}) for converting between base-10 (Decimal) and base-12 (Dozenal).
  \item[Services:] Converts a decimal number to a Dozenal string. Converts a Dozenal string to a decimal number. Used for scoring and UI display.
  \item[Implemented By:] The Crazy Four (TypeScript)
  \item[State Variables:] None (pure functions).
  \item[Access Programs:] `toDozenal(decimalValue)` $\to$ `string`; `toDecimal(dozenalString)` $\to$ `number`
  \item[Exceptions:] `InvalidBaseString`
\end{description}
\end{quote}

\subsubsection{Game Actions Module (\mref{mGameActions})}
\begin{quote}
\begin{description}
  \item[Secrets:] The data structure and serialization format for all player actions sent between client and server.
  \item[Services:] Provides type-safe action constructors, validation, and JSON serialization/deserialization for network transmission.
  \item[Implemented By:] The Crazy Four (TypeScript - shared code between frontend and backend)
  \item[State Variables:] None (stateless utility module).
  \item[Access Programs:] 
    \begin{itemize}
      \item \texttt{createPlayCardAction(playerID, cardID)} $\to$ \texttt{PlayCardAction}
      \item \texttt{createDrawCardAction(playerID)} $\to$ \texttt{DrawCardAction}
      \item \texttt{createDeclareSuitAction(playerID, suit)} $\to$ \texttt{DeclareSuitAction}
      \item \texttt{createScoreTallyAction(playerID, remainingCards)} $\to$ \texttt{ScoreTallyAction}
      \item \texttt{validateAction(action)} $\to$ \texttt{boolean}
      \item \texttt{serializeAction(action)} $\to$ \texttt{JSON}
      \item \texttt{deserializeAction(json)} $\to$ \texttt{GameAction}
    \end{itemize}
  \item[Exceptions:] \texttt{InvalidCard}, \texttt{InvalidSuit}, \texttt{SerializationError}, \texttt{DeserializationError}
\end{description}
\end{quote}

\subsection{Software Decision Module - Backend}

\subsubsection{API Module (\mref{mAPI})}
\begin{quote}
\begin{description}
  \item[Secrets:] REST API route definitions, request/response formats (JSON), HTTP status codes.
  \item[Services:] Provides stateless HTTP endpoints for user authentication (FR-10..13), profile management (FR-15..17), and game creation.
  \item[Implemented By:] The Crazy Four (Node.js, Express)
  \item[State Variables:] None (stateless).
  \item[Access Programs:] `POST /api/auth/signup`, `POST /api/auth/login`, `GET /api/profile`, `PUT /api/profile`, `DELETE /api/profile`
  \item[Exceptions:] `AuthError` (401/403), `ValidationError` (400), `NotFound` (404)
\end{description}
\end{quote}

\subsubsection{Real-time Gateway Module (\mref{mRealtimeGateway})}
\begin{quote}
\begin{description}
  \item[Secrets:] WebSocket protocol, message serialization, room/session management. Server-authoritative state synchronization logic.
  \item[Services:] Manages active game sessions, broadcasts game state updates to clients, receives and validates moves from clients in real-time.
  \item[Implemented By:] The Crazy Four (Node.js, Socket.io)
  \item[State Variables:] `activeGames`: Map<`GameID`, `GameSession`>
  \item[Access Programs:] `on('connection', socket)`; `on('playCard', action)`; `on('drawCard', action)`; `on('declareSuit', action)`; `on('submitScoreTally', action)`; `emit('gameStateUpdate', state)`; `emit('scoreTallyConfirmed', summary)`
  \item[Exceptions:] `InvalidMove`, `NotYourTurn`, `SessionError`
\end{description}
\end{quote}

\subsubsection{Matchmaking Module (\mref{mMatchmaking})}
\begin{quote}
\begin{description}
  \item[Secrets:] Logic for pairing players, managing game lobbies, and handling invites.
  \item[Services:] Allows users to create, join, or be matched into a game session.
  \item[Implemented By:] The Crazy Four (Node.js)
  \item[State Variables:] `lobbies`: Map<`LobbyID`, `Lobby`>
  \item[Access Programs:] `createLobby(userID)` $\to$ `LobbyID`; `joinLobby(lobbyID, userID)` $\to$ `void`
  \item[Exceptions:] `LobbyFull`, `LobbyNotFound`
\end{description}
\end{quote}

\subsubsection{Authentication Module (\mref{mAuth})}
\begin{quote}
\begin{description}
  \item[Secrets:] Password hashing algorithm (e.g., bcrypt), salt generation, JWT/session token structure and secret key.
  \item[Services:] Handles user account creation, validates credentials, manages guest sessions, issues and verifies session tokens.
  \item[Implemented By:] The Crazy Four (Node.js)
  \item[State Variables:] None.
  \item[Access Programs:] `registerUser(username, password)` $\to$ `User`; `loginUser(username, password)` $\to$ `Token`; `verifyToken(token)` $\to$ `UserID`
  \item[Exceptions:] `InvalidCredentials`, `UserExists`, `TokenExpired`
\end{description}
\end{quote}

\subsubsection{Repository Module (\mref{mRepository})}
\begin{quote}
\begin{description}
  \item[Secrets:] Database schema (tables, columns, relations), SQL queries, connection pooling.
  \item[Services:] Provides an interface for all data persistence. Stores and retrieves user data (FR-14, FR-15), game history, and scores. Handles updates (FR-16) and deletions (FR-17).
  \item[Implemented By:] The Crazy Four (Node.js, node-postgres)
  \item[State Variables:] Database connection pool.
  \item[Access Programs:] `findUserByUsername(username)`, `createUser(data)`, `saveGameResult(result)`, `getUserProfile(userID)`, `updateUserProfile(userID, data)`, `deleteUser(userID)`
  \item[Exceptions:] `DatabaseConnectionError`, `RecordNotFound`, `UniqueConstraintViolation`
\end{description}
\end{quote}

\subsubsection{Audit Module (\mref{mAudit})}
\begin{quote}
\begin{description}
  \item[Secrets:] Log format, log storage location, retention policy.
  \item[Services:] Logs important system events (e.g., login, game end, errors) for debugging and security auditing.
  \item[Implemented By:] The Crazy Four (Node.js, Winston)
  \item[State Variables:] Logger instance.
  \item[Access Programs:] `log.info(message)`, `log.warn(message)`, `log.error(message)`
  \item[Exceptions:] `LogWriteError`
\end{description}
\end{quote}

\subsection{Software Decision Module - Frontend}

\subsubsection{Real-time Client Module (\mref{mRealtimeClient})}
\begin{quote}
\begin{description}
  \item[Secrets:] WebSocket connection state, reconnection logic.
  \item[Services:] Connects to the \mref{mRealtimeGateway}, sends user moves, and receives game state updates, applying them to the client-side state.
  \item[Implemented By:] The Crazy Four (TypeScript, Socket.io-client)
  \item[State Variables:] `socket`: Socket, `isConnected`: bool
  \item[Access Programs:] `connect()`; `disconnect()`; `emit('playCard', action)`; `emit('drawCard')`; `emit('declareSuit', action)`; `emit('submitScoreTally', action)`; `on('gameStateUpdate', callback)`; `on('scoreTallyConfirmed', callback)`
  \item[Exceptions:] `ConnectionFailed`
\end{description}
\end{quote}

\subsubsection{Application Shell Module (\mref{mAppShell})}
\begin{quote}
\begin{description}
  \item[Secrets:] Application routing table, global layout structure (header, footer), global state management (e.g., auth status).
  \item[Services:] Renders the main application layout, controls page navigation, and manages global UI state.
  \item[Implemented By:] The Crazy Four (React)
  \item[State Variables:] `currentUser`, `currentRoute`
  \item[Access Programs:] (Rendered React component)
  \item[Exceptions:] `RouteNotFound`
\end{description}
\end{quote}

\subsubsection{Authentication Client Module (\mref{mAuthClient})}
\begin{quote}
\begin{description}
  \item[Secrets:] How auth tokens are stored (e.g., localStorage, httpOnly cookie).
  \item[Services:] Provides UI components (login/signup forms) and client-side logic for authentication. Communicates with \mref{mAPI}.
  \item[Implemented By:] The Crazy Four (React)
  \item[State Variables:] `username`, `password`, `isLoading`: bool, `error`: string
  \item[Access Programs:] `handleLogin()`, `handleSignup()`, `handleLogout()`
  \item[Exceptions:] `AuthUIError`
\end{description}
\end{quote}

\subsubsection{Lobby View Module (\mref{mLobbyView})}
\begin{quote}
\begin{description}
  \item[Secrets:] UI for listing, creating, and joining games.
  \item[Services:] Renders the game lobby, interacts with \mref{mAPI} and \mref{mRealtimeClient} to manage matchmaking.
  \item[Implemented By:] The Crazy Four (React)
  \item[State Variables:] `lobbiesList`: [ ], `selectedLobby`
  \item[Access Programs:] `handleCreateGame()`, `handleJoinGame(lobbyID)`
  \item[Exceptions:] None
\end{description}
\end{quote}

\subsubsection{Game Board View Module (\mref{mGameBoardView})}
\begin{quote}
\begin{description}
  \item[Secrets:] The DOM/CSS for rendering cards, hands, and the board. Animation logic.
  \item[Services:] Renders the main game interface (player hand, discard pile, deck). Visually highlights valid moves (FR-9).
  \item[Implemented By:] The Crazy Four (React)
  \item[State Variables:] `clientGameState`, `validMoves`: [ ]
  \item[Access Programs:] (Rendered React component based on props)
  \item[Exceptions:] None
\end{description}
\end{quote}

\subsubsection{Move Controller Module (\mref{mMoveController})}
\begin{quote}
\begin{description}
  \item[Secrets:] Client-side input handling logic (click, drag-and-drop).
  \item[Services:] Captures user input (e.g., clicking a card), performs client-side pre-validation, and submits the move to the \mref{mRealtimeClient}.
  \item[Implemented By:] The Crazy Four (React hooks/event handlers)
  \item[State Variables:] `selectedCard`
  \item[Access Programs:] `requestPlayCard(cardID)`; `requestDrawCard()`; `requestSuitChange(suit)`; `requestScoreTally(remainingCards)`
  \item[Exceptions:] `InvalidMoveUI` (provides polite feedback)
\end{description}
\end{quote}

\subsubsection{Scoreboard View Module (\mref{mScoreboardView})}
\begin{quote}
\begin{description}
  \item[Secrets:] UI layout for displaying end-of-game or end-of-round scores.
  \item[Services:] Renders the score, clearly displaying both Decimal and Dozenal values (FR-7).
  \item[Implemented By:] The Crazy Four (React)
  \item[State Variables:] `scoreData`
  \item[Access Programs:] (Rendered React component based on props)
  \item[Exceptions:] None
\end{description}
\end{quote}

\subsubsection{Profile View Module (\mref{mProfileView})}
\begin{quote}
\begin{description}
  \item[Secrets:] UI layout for displaying game history and user statistics.
  \item[Services:] Renders the user's profile and game history (FR-15). Allows user to request data deletion (FR-17).
  \item[Implemented By:] The Crazy Four (React)
  \item[State Variables:] `profileData`, `gameHistory`: [ ]
  \item[Access Programs:] `handleDeleteAccount()`
  \item[Exceptions:] None
\end{description}
\end{quote}

\section{Traceability Matrix} \label{SecTM}

This section shows two traceability matrices: between the modules and the
requirements and between the modules and the anticipated changes.

% the table should use mref, the requirements should be named, use something
% like fref
\begin{table}[H]
\centering
\caption{Trace Between Requirements and Modules (TblRT)}
\label{TblRT}
\begin{tabularx}{\textwidth}{lX}
\toprule
\textbf{Requirement (FR/NFR/SR)} & \textbf{Primary Modules}\\
\midrule
FR-1 Start new game & \mref{mAPI}, \mref{mMatchmaking}, \mref{mGameEngine}, \mref{mRepository}\\
FR-2 Turn management & \mref{mGameEngine}, \mref{mRules}, \mref{mGameActions}, \mref{mRealtimeGateway}, \mref{mRealtimeClient}, \mref{mMoveController}, \mref{mGameBoardView}\\
FR-3 Rule validation & \mref{mRules}, \mref{mGameEngine}, \mref{mGameActions}, \mref{mMoveController}, \mref{mGameBoardView}\\
FR-4 Special cards & \mref{mRules}, \mref{mGameEngine}, \mref{mGameActions}, \mref{mMoveController}, \mref{mGameBoardView}\\
FR-5 End of game & \mref{mGameEngine}, \mref{mScoring}, \mref{mScoreboardView}, \mref{mRepository}\\
FR-6 Calculate score & \mref{mScoring}, \mref{mBaseConversion}, \mref{mGameActions}, \mref{mScoreboardView}\\
FR-7 Display score & \mref{mScoreboardView}, \mref{mBaseConversion}\\
FR-9 Highlight valid moves & \mref{mMoveController}, \mref{mGameBoardView}, \mref{mRules}\\
FR-10 Account creation & \mref{mAPI}, \mref{mAuth}, \mref{mRepository}, \mref{mAuthClient}\\
FR-11 Login or Logout & \mref{mAPI}, \mref{mAuth}, \mref{mRepository}, \mref{mAuthClient}\\
FR-12 Guest mode & \mref{mAPI}, \mref{mAuth}, \mref{mAuthClient}\\
FR-13 Credential validation & \mref{mAuth}, \mref{mAPI}, \mref{mRepository}\\
FR-14 Data storage & \mref{mRepository}, \mref{mAPI}, \mref{mAudit}\\
FR-15 Data retrieval & \mref{mRepository}, \mref{mAPI}, \mref{mProfileView}\\
FR-16 Data update & \mref{mRepository}, \mref{mAPI}, \mref{mProfileView}\\
FR-17 Data deletion & \mref{mRepository}, \mref{mAPI}, \mref{mProfileView}\\
\midrule
NFR (Performance) & \mref{mRealtimeGateway}, \mref{mGameEngine}, \mref{mRealtimeClient}, \mref{mGameBoardView}\\
NFR (Usability) & \mref{mGameBoardView}, \mref{mAppShell}\\
NFR (Robustness) & \mref{mRealtimeClient}, \mref{mRealtimeGateway}, \mref{mRepository}\\
NFR (Maintainability) & \mref{mRules}, \mref{mScoring}, \mref{mAPI}, \mref{mAudit}\\
\midrule
SR-1 (Dozenal validation) & \mref{mRules}, \mref{mBaseConversion}, \mref{mGameEngine}\\
SR-2 (UI feedback) & \mref{mGameBoardView}, \mref{mAppShell}\\
SR-3 (Data persistence) & \mref{mRepository}, \mref{mAPI}\\
SR-4 (Accurate scoring) & \mref{mScoring}, \mref{mBaseConversion}\\
SR-5 (Session recovery) & \mref{mRealtimeClient}, \mref{mRealtimeGateway}, \mref{mAuth}\\
SR-7 (Encrypted transmit) & (Hardware-Hiding: TLS Layer), \mref{mAPI}, \mref{mRealtimeGateway}\\
SR-8 (Secure storage) & \mref{mRepository}, \mref{mAuth} \\
SR-10 (Input validation) & \mref{mAPI}, \mref{mRealtimeGateway}, \mref{mMoveController}\\
\bottomrule
\end{tabularx}
\end{table}

\begin{table}[H]
\centering
\caption{Trace Between Anticipated Changes and Modules (TblACT)}
\label{TblACT}
\begin{tabularx}{\textwidth}{p{0.2\textwidth} X}
\toprule
\textbf{AC} & \textbf{Modules}\\
\midrule
\acref{acRules} & \mref{mRules}, \mref{mGameEngine} \\
\acref{acBases} & \mref{mBaseConversion}, \mref{mScoring}, \mref{mRules}, \mref{mScoreboardView} \\
\acref{acScoring} & \mref{mScoring}, \mref{mScoreboardView} \\
\acref{acUI} & \mref{mAppShell}, \mref{mGameBoardView}, \mref{mScoreboardView}, \mref{mLobbyView}, \mref{mProfileView} \\
\acref{acSchema} & \mref{mRepository}, \mref{mAPI}, \mref{mProfileView} \\
\acref{acRealtime} & \mref{mRealtimeGateway}, \mref{mRealtimeClient} \\
\acref{acAuth} & \mref{mAuth}, \mref{mAuthClient}, \mref{mAPI} \\
\bottomrule
\end{tabularx}
\end{table}
\section{Use Hierarchy Between Modules} \label{SecUse}

In this section, the uses hierarchy between modules is
provided. \citet{Parnas1978} said of two programs A and B that A {\em uses} B if
correct execution of B may be necessary for A to complete the task described in
its specification. That is, A {\em uses} B if there exist situations in which
the correct functioning of A depends upon the availability of a correct
implementation of B.  Figure \ref{FigUH} illustrates the use relation between
the modules. It can be seen that the graph is a directed acyclic graph
(DAG). Each level of the hierarchy offers a testable and usable subset of the
system, and modules in the higher level of the hierarchy are essentially simpler
because they use modules from the lower levels.

\wss{The uses relation is not a data flow diagram.  In the code there will often
be an import statement in module A when it directly uses module B.  Module B
provides the services that module A needs.  The code for module A needs to be
able to see these services (hence the import statement).  Since the uses
relation is transitive, there is a use relation without an import, but the
arrows in the diagram typically correspond to the presence of import statement.}

\wss{If module A uses module B, the arrow is directed from A to B.}

\begin{figure}[H]
\centering
%\includegraphics[width=0.7\textwidth]{UsesHierarchy.png}
\caption{Use hierarchy among modules}
\label{FigUH}
\end{figure}

%\section*{References}

\section{User Interfaces}

\wss{Design of user interface for software and hardware.  Attach an appendix if
needed. Drawings, Sketches, Figma}

\section{Design of Communication Protocols}

\wss{If appropriate}

\section{Timeline}

\wss{Schedule of tasks and who is responsible}

\wss{You can point to GitHub if this information is included there}

\bibliographystyle {plainnat}
\bibliography{../../../refs/References}

\newpage{}

\end{document}