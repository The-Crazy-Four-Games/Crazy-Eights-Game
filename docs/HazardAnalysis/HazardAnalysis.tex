\documentclass{article}

\usepackage{booktabs}
\usepackage{tabularx}
\usepackage{hyperref}
\usepackage{booktabs,longtable,tabularx,ragged2e}
\usepackage{changepage}
\usepackage{tabularx}  % for flexible-width tables
\usepackage{booktabs}  % for \toprule, \midrule, \bottomrule
\usepackage{float}     % for [H] placement specifier

\hypersetup{
    colorlinks=true,       % false: boxed links; true: colored links
    linkcolor=red,          % color of internal links (change box color with linkbordercolor)
    citecolor=green,        % color of links to bibliography
    filecolor=magenta,      % color of file links
    urlcolor=cyan           % color of external links
}

\title{Hazard Analysis\\\progname}

\author{
    Team \#25, The Crazy Four \\[1ex]
    Ruida Chen \\
    Ammar Sharbat \\
    Alvin Qian \\
    Jiaming Li
}

\date{}

\input{../Comments}
\input{../Common}

\begin{document}

\maketitle
\thispagestyle{empty}

~\newpage

\pagenumbering{roman}

\begin{table}[hp]
\caption{Revision History} \label{TblRevisionHistory}
\begin{tabularx}{\textwidth}{llX}
\toprule
\textbf{Date} & \textbf{Developer(s)} & \textbf{Change}\\
\midrule
10.9 & Jiaming Li & FMEA\\
10.9 & Ruida Chen & Introduction, Scope and Purpose, Components, Reflection Q4\\
10.9 & Alvin Qian & Added Assumptions\\
... & ... & ...\\
\bottomrule
\end{tabularx}
\end{table}

~\newpage

\tableofcontents

~\newpage

\pagenumbering{arabic}


\section{Introduction}

For a card game that entertains users by applying different game rules in different numerical systems to teach various number bases, reliability of the software and the user safety are neccessary attributes. In this context, a hazard is defined as any situation within the system or in the surrounding environment that may cause harm, such as confusion, data loss or resulting in incorrect learning outcomes. In this project, hazards are not limited to physical concerns; they mainly focus on software-related risks. These issues may cause users to become frustrated or reduce the educational value of the game. The Failure Modes and Effects Analysis (FMEA) method is used to identify and evaluate these possible failure modes within key components, guiding the team to improve system fairness, learning accuracy, and overall robustness.

\section{Scope and Purpose of Hazard Analysis}

The purpose of this hazard analysis is to identify components of the card game system that may lead to undesirable outcomes and to understand the potential losses that could occur if such hazards are realized. These losses may include:
\begin{itemize}
	\item Loss of learning effectiveness, where players receive incorrect results or misleading explanations about numeral bases.
	\item Loss of user trust or engagement, caused by unfair scoring, interface errors, or game desynchronization.
	\item Loss of data integrity, resulting in corrupted game records or incorrect user progress tracking.
	\item Loss of system availability, due to server failure or communication breakdowns during active sessions.
\end{itemize}
The analysis considers both situations when the system functions incorrectly and when it behaves as designed but under unexpected conditions (e.g., network lag or user misuse). By identifying potential hazards early, this document supports the design of a more stable, accurate, and user-centered learning experience.

\section{System Boundaries and Components}
\subsection{Frontend}
This component provides the main interface for players to interact with the system, it captures user inputs such as playing, drawing, or skipping cards and provides visual feedback and animations during gameplay.
\subsection{Backend Server}
This component manages the overall system logic, coordinating user sessions, handle requests, and enforcing gameplay.
\subsection{Database}
This component stores persistent data such as player account information, game history, and user configuration settings.
\subsection{Real-Time Communication Layer}
This component maintains live connections between clients and the server to synchronize player actions in real time.
\subsection{Game Rule Module}
This component enforces the game's core rules, such as base conversion calculations, card validation, and scoring.
\subsection{Auth \& User Management System}
This component handles user registration, login, and session verification.
\subsection{External Libraries and APIs}
This component relies on several open-source frameworks and libraries.

\section{Critical Assumptions}

The assumptions below focus on a realistic deployment and user population. 
\subsection*{Assumptions deliberately not made}
\begin{itemize}
  \item We do not assume client input is trusted. All moves are validated on the server.
  \item We do not assume stable connectivity for the entire session. Rejoin and reconciliation are required features.
  \item We do not assume prior knowledge of base 12. In app explanations and hints are mandatory.
\end{itemize}

\begin{table}[H]
\caption{Critical Assumptions and Implications}\label{tab:assumptions}
\begin{tabularx}{\textwidth}{lX X X}
\toprule
\textbf{ID} & \textbf{Assumption} & \textbf{Rationale} & \textbf{If violated} \\
\midrule
A1 & Users access the game on a supported desktop or laptop browser (Chrome, Firefox, Edge) with JavaScript, cookies, and local storage enabled. & Limits test matrix and aligns with target environment. & UI features malfunction, loss of state persistence, inaccessible components. \\
A2 & Average network latency under 300 ms and packet loss under 1\% during play. & Real time turns and fairness depend on timely messages. & Turn desync, duplicate actions, stale state decisions. \\
A3 & Server and database share a synchronized time source via NTP and timeouts use a monotonic clock. & Consistent session expiry and replay ordering. & Expired tokens accepted, rejoin windows miscalculated, audit confusion. \\
A4 & WebSocket over TLS 1.2 or higher is available from client networks. & Required for live play and low latency. & Falls back to polling or blocks play which raises latency and server load. \\
A5 & Database durability is at least write ahead logging with fsync on commit. & Protects match history and user settings. & Corrupted or missing results after crashes or restarts. \\
A6 & No personally sensitive data beyond username and optional email is stored and there are no financial transactions. & Reduces compliance and breach impact. & If sensitive data is later added, privacy risk and legal exposure increase. \\
A7 & Client devices provide basic accessibility aids such as keyboard, focus navigation, and screen reader support. & Meets usability and accessibility requirements. & Some users cannot operate the game which creates rule misunderstanding hazards. \\
\bottomrule
\end{tabularx}
\end{table}

\section{Failure Mode and Effect Analysis}
\setlength{\LTleft}{-3cm}
\setlength{\LTright}{0pt}
    \begin{small}
        \begin{longtable}{@{}p{2.6cm} p{3.2cm} p{3.0cm} p{3.0cm} p{6.0cm} p{1.2cm} p{1.2cm}@{}}
            \caption{Failure Mode and Effect Analysis (FMEA)}\label{tab:fmea}\\
            \toprule
            \textbf{Component} & \textbf{Failure Modes} & \textbf{Effects of Failure} &
            \textbf{Causes of Failure} & \textbf{Recommended Action} & \textbf{SR} & \textbf{Ref.}\\
            \midrule
            \endfirsthead
            \toprule
            \textbf{Component} & \textbf{Failure Modes} & \textbf{Effects of Failure} &
            \textbf{Causes of Failure} & \textbf{Recommended Action} & \textbf{SR} & \textbf{Ref.}\\
            \midrule
            \endhead
            \bottomrule
            \endfoot


            Game Engine &
            Incorrect Dozenal validation (e.g., misinterpreting sum = 12). &
            Invalid moves; breaks gameplay logic and educational purpose. &
            Logic bug; base–10/12 conversion error. &
            (a) Add unit tests for all Dozenal comparisons; (b) property-based tests for edge cases. &
            SR-1 & H1-1 \\ \midrule

            UI Layer &
            Hints/feedback not displayed during gameplay. &
            Player confusion; reduced educational value. &
            Rendering delay; missing event callback. &
            (a) Async UI event queue with logging; (b) fallback text-only hint when animation fails. &
            SR-2 & H1-2 \\ \midrule

            Login Manager &
            Player unable to log in or session lost. &
            Loss of access to saved data or multiplayer mode. &
            Session timeout; invalid credential cache. &
            (a) Guest-mode fallback; (b) local session cache; retry with exponential backoff. &
            SR-3 & H1-3 \\ \midrule

            Data Manager &
            Save corruption during file write. &
            Loss of player profile or progress. &
            Non-atomic save; unexpected shutdown. &
            (a) Atomic writes + checksum on load; (b) periodic backups and restore flow. &
            SR-4 & H1-4 \\ \midrule

            Score Calculator &
            Wrong Dozenal–decimal conversion or totals. &
            Inconsistent scores; undermines trust. &
            Arithmetic bug; type overflow. &
            (a) Secondary calculator cross-check; (b) property-based tests on random inputs. &
            SR-5 & H1-5 \\ \midrule

            Network (future) &
            Desynchronized multiplayer state. &
            Different states across clients. &
            Packet loss; high latency. &
            (a) State-hash verification; (b) timeout-based resync; authoritative host. &
            SR-6 & H1-6 \\ \midrule

            Educational Feedback &
            Incorrect/misleading Dozenal explanation. &
            Players learn wrong concepts. &
            Faulty rule mapping; stale content. &
            (a) Peer-reviewed hint knowledge base; (b) versioning and CI checks. &
            SR-7 & H1-7 \\ \midrule

            Security Layer &
            Unencrypted data at rest/in transit. &
            Privacy breach; data leak. &
            Missing encryption; insecure local storage. &
            (a) AES-256 at rest; (b) HTTPS/TLS in transit; (c) secrets in OS keystore. &
            SR-8 & H1-8 \\
        \end{longtable}
    \end{small}

\section{Safety and Security Requirements}

\wss{Newly discovered requirements.  These should also be added to the SRS.  (A
rationale design process how and why to fake it.)}

\section{Roadmap}

\wss{Which safety requirements will be implemented as part of the capstone timeline?
Which requirements will be implemented in the future?}

\newpage{}

\section*{Appendix --- Reflection}

\wss{Not required for CAS 741}

\input{../Reflection.tex}

\begin{enumerate}
    \item What went well while writing this deliverable? 
    \item What pain points did you experience during this deliverable, and how
    did you resolve them?
    \item Which of your listed risks had your team thought of before this
    deliverable, and which did you think of while doing this deliverable? For
    the latter ones (ones you thought of while doing the Hazard Analysis), how
    did they come about?
    \paragraph{Answer:}
    Before this deliverable, our team had already considered general risks such as scheduling conflicts, Git merge issues, and limited testing time near deadlines.
    During the Hazard Analysis, we identified new, more specific risks, including Dozenal validation logic errors, data corruption during saving, and inaccurate educational feedback. These emerged as we analyzed each system component in the FMEA table, which led us to think more deeply about technical failures and their impact on users.
    \item Other than the risk of physical harm (some projects may not have any
    appreciable risks of this form), list at least 2 other types of risk in
    software products. Why are they important to consider? 
	\begin{itemize}
	\item Usability and human-error risks: Poorly designed user interfaces, low-quality GUI animation, unclear instructions can cause user to make mistake or misunderstand the system, these risks are important because they directly affect user satisfaction, learning outcomes and overall quality of the software.
	\item Data integrity and privacy risks: These occur when user data is lost, corrupted or exposed due to software defects, inappropriate authentication. Such risks are critical because they can lead to permanent loss of user information and legal issues, breaches can severely damage credibility and violate data protection regulations.
	
	\end{itemize}
\end{enumerate}

\end{document}