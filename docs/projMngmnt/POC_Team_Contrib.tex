\documentclass{article}

\usepackage{float}
\restylefloat{table}

\usepackage{booktabs}

\title{Team Contributions: POC\\The Crazy Tens}

\author{
    Team \#25, The Crazy Four \\[1ex]
    Ruida Chen \\
    Ammar Sharbat \\
    Alvin Qian \\
    Jiaming Li
}

\date{\today}

\input{../Comments}
\input{../Common}

\begin{document}

\maketitle

This document summarizes the contributions of each team member up to the POC
Demo.  The time period of interest is the time between the beginning of the term
and the POC demo.

\section{Demo Plans}

\subsection*{Overview}
The POC demo will be performed in-person and run locally from a group member's laptop. The goal is to demonstrate the core gameplay mechanics and an end-to-end 1v1 game flow.

\subsection*{Setup}
\begin{itemize}
	\item Start the local server (\texttt{Node.js} terminal server) on the presenter's machine.
	\item Use the hard-coded two-player configuration (no authentication for the POC).
	\item Open the game in a browser on the presenter machine.
\end{itemize}

\subsection*{Demo flow (approx. 5--7 minutes)}
\begin{enumerate}
	\item Brief introduction of demo objective (30s).
	\item Show initial game state: deck creation and hands dealt to both players.
	\item Two team members will control each player and demonstrate gameplay:
		\begin{itemize}
			\item Playing cards from a hand.
			\item Demonstrating matching logic (suit, rank, and add-to-10 rule).
			\item Demonstrating wildcard logic (changing the suit).
			\item Demonstrating drawing a card when no valid plays are available.
		\end{itemize}
	\item Drive the game to an endgame state and show the end condition.
	\item Conclude with limitations and next steps, then take questions.
\end{enumerate}

\subsection*{Notes}
\begin{itemize}
	\item No login/authentication is required for this POC demo.
	\item Will mention current limitations (hard-coded users, basic UI) and planned next steps.
\end{itemize}


\section{Team Meeting Attendance}

\begin{table}[H]
\centering
\begin{tabular}{ll}
\toprule
\textbf{Student} & \textbf{Meetings}\\
\midrule
Total & 3 \\
Ruida Chen & 3 \\
Jiaming Li & 3 \\
Alvin Qian & 3 \\
Ammar Sharbat & 3 \\
\bottomrule
\end{tabular}
\end{table}

Explanation: \\\\
Realistically, we have had very few team meetings. This is largely because of the packed schedules of 3 team members and also the fact that project management and communication through Discord has taken precedence over in person or virtual meetings.\\\\
No team meetings above have a corresponding GitHub issue just yet, as we were not aware about making one before said meetings. We plan to create these issues of old for traceability. Also, moving forward we will proactively make issues with agendas for all team meetings.

\section{Supervisor/Stakeholder Meeting Attendance}

\noindent \textbf{Supervisor's Name: } Paul Rapoport; Email: rapoport@mcmaster.ca

\begin{table}[H]
\centering
\begin{tabular}{ll}
\toprule
\textbf{Student} & \textbf{Meetings}\\
\midrule
Total & 7 \\
Ruida Chen & 4 \\
Jiaming Li & 3 \\
Alvin Qian & 4 \\
Ammar Sharbat & 8 \\
\bottomrule
\end{tabular}
\end{table}

Explanation:\\\\
The last 2 supervisor meetings are on Github (Issues #43 and #82).\\\\
Many meeting with Professor Rapoport were initiated by teammate Ammar Sharbat, because he is the team liaison to the supervisor. He also has taken a keen interest in the Core Problem of Game Design and Game Mechanics, and has met with the professor several times to discuss issues pertaining to these topics.\\\\
Outside of the very very project search meeting (which was initiated by another teammate), teammate Ammar has also initiated all group meetings with the supervisor.\\\\
Other teammates have attended important meetings to update and discuss project progress and problems with the supervisor. Teammate Jiaming Li missed one meeting due to family reasons.

\section{Lecture Attendance}

\wss{For each team member how many lectures have they attended over the time
period of interest.  This number should be determined from the lecture issues in
the team's repo. You can find the number of lectures in the time period of
interest by looking at the
\href{https://calendar.google.com/calendar/u/0/embed?src=rnboqiaki1k2la7rpu3bn0um58@group.calendar.google.com&ctz=America/Toronto}
{Google calendar} for the capstone course.}

\wss{NOTE: There will be approximately 13 lectures between the start of class
and the POC demos}

\begin{table}[H]
\centering
\begin{tabular}{ll}
\toprule
\textbf{Student} & \textbf{Lectures}\\
\midrule
Total & Num\\
Name 1 & Num\\
Name 2 & Num\\
Name 3 & Num\\
Name 4 & Num\\
Name 5 & Num\\
\bottomrule
\end{tabular}
\end{table}

\wss{If needed, an explanation for the lecture attendance can be provided here.}

\section{TA Document Discussion Attendance}

\wss{For each team member how many of the informal document discussion meetings
with the TA were attended over the time period of interest.}

\noindent \textbf{TA's Name: } [fill in this information]

\begin{table}[H]
\centering
\begin{tabular}{ll}
\toprule
\textbf{Student} & \textbf{Lectures}\\
\midrule
Total & Num\\
Name 1 & Num\\
Name 2 & Num\\
Name 3 & Num\\
Name 4 & Num\\
Name 5 & Num\\
\bottomrule
\end{tabular}
\end{table}

\wss{If needed, an explanation for the attendance can be provided here.}

\section{Commits}

\begin{table}[H]
\centering
\begin{tabular}{lll}
\toprule
\textbf{Student} & \textbf{Commits} & \textbf{Percent}\\
\midrule
Total & 97 & 100\% \\
Ruida Chen & 25 & 25.8\% \\
Jiaming Li & 26 & 26.8\% \\
Alvin Qian & 24 & 24.7\% \\
Ammar Sharbat & 22 & 22.7\% \\
\bottomrule
\end{tabular}
\end{table}

\section{Issue Tracker}

\begin{table}[H]
\centering
\begin{tabular}{lll}
\toprule
\textbf{Student} & \textbf{Authored (O+C)} & \textbf{Assigned (C only)}\\
\midrule
Ruida Chen & 19 & 9 \\
Jiaming Li & 17 & 7 \\
Alvin Qian & 19 & 8 \\
Ammar Sharbat & 24 & 9 \\
\bottomrule
\end{tabular}
\end{table}

\section{CICD}

The project repository is hosted on GitHub and uses GitHub Actions for Continuous Integration and Continuous Deployment (CICD).
Each push or pull request triggers an automated workflow that performs the following tasks:

\begin{itemize}
    \item \textbf{Build and Lint:} The workflow installs all dependencies, compiles the code, and runs ESLint to enforce consistent formatting and syntax.
    \item \textbf{Unit Testing:} All Jest test suites are executed automatically. Code coverage reports are uploaded to Codecov.
    \item \textbf{Static Analysis:} CodeQL is run to detect potential vulnerabilities and logic errors.
    \item \textbf{Artifact Packaging:} For successful builds, the workflow produces a testable web or desktop artifact for internal review.
\end{itemize}

This setup ensures that any code merged into the \texttt{main} branch has passed validation for correctness, maintainability, and security.
By automating these checks, CICD reduces integration errors and accelerates the development feedback cycle.

\section{Team Charter Trigger Items}

The team has identified several triggers within the team charter to monitor collaboration and performance consistency:

\begin{itemize}
    \item \textbf{Commit Frequency:} Each member should contribute at least one meaningful commit per week.
    Falling below this threshold for two consecutive weeks triggers a discussion about workload balance.
    \item \textbf{Meeting Attendance:} Missing two consecutive team meetings without prior notice triggers a check-in with the member to identify scheduling or communication issues.
    \item \textbf{Branch Discipline:} All code changes must go through a pull request reviewed by at least one teammate.
    Direct commits to \texttt{main} are not allowed and will trigger an immediate process review.
    \item \textbf{Responsiveness:} Team members are expected to reply to key project communications (e.g., PR reviews or Slack updates) within 24 hours.
    Failure to respond repeatedly triggers a group discussion for reassigning responsibilities.
\end{itemize}

So far, no major trigger violations have occurred. The team has maintained consistent communication and review discipline.
If violations are observed in the future, the plan is to (1) hold a brief retrospective discussion,
(2) revise or clarify the trigger threshold if needed, and (3) document the agreed corrective action in the next meeting notes.


\section{Additional Productivity Metrics}

\wss{If your team has additional metrics of productivity, please feel free to
add them to this report.}

\end{document}