\documentclass{article}

\usepackage{booktabs}
\usepackage{tabularx}

\title{Development Plan\\The Crazy Eights}

\author{Ruida Chen \\ Alvin Qian \\ Ammar, Sharbat \\ Jiaming Li}


\date{September 2025}

%% Comments

\usepackage{color}

\newif\ifcomments\commentstrue %displays comments
%\newif\ifcomments\commentsfalse %so that comments do not display

\ifcomments
\newcommand{\authornote}[3]{\textcolor{#1}{[#3 ---#2]}}
\newcommand{\todo}[1]{\textcolor{red}{[TODO: #1]}}
\else
\newcommand{\authornote}[3]{}
\newcommand{\todo}[1]{}
\fi

\newcommand{\wss}[1]{\authornote{magenta}{SS}{#1}} 
\newcommand{\plt}[1]{\authornote{cyan}{TPLT}{#1}} %For explanation of the template
\newcommand{\an}[1]{\authornote{cyan}{Author}{#1}}

%% Common Parts

\newcommand{\progname}{ProgName} % PUT YOUR PROGRAM NAME HERE
\newcommand{\authname}{Team \#, Team Name
\\ Student 1 name
\\ Student 2 name
\\ Student 3 name
\\ Student 4 name} % AUTHOR NAMES                  

\usepackage{hyperref}
    \hypersetup{colorlinks=true, linkcolor=blue, citecolor=blue, filecolor=blue,
                urlcolor=blue, unicode=false}
    \urlstyle{same}
                                


\begin{document}

\maketitle

\begin{table}[hp]
\caption{Revision History} \label{TblRevisionHistory}
\begin{tabularx}{\textwidth}{llX}
\toprule
\textbf{Date} & \textbf{Developer(s)} & \textbf{Change}\\
\midrule
Date1 & Name(s) & Description of changes\\
Date2 & Name(s) & Description of changes\\
... & ... & ...\\
\bottomrule
\end{tabularx}
\end{table}

\newpage{}

\wss{Put your introductory blurb here.  Often the blurb is a brief roadmap of
what is contained in the report.}

\wss{Additional information on the development plan can be found in the
\href{https://gitlab.cas.mcmaster.ca/courses/capstone/-/blob/main/Lectures/L02b_POCAndDevPlan/POCAndDevPlan.pdf?ref_type=heads}
{lecture slides}.}

\section{Confidential Information?}

\wss{State whether your project has confidential information from industry, or
not.  If there is confidential information, point to the agreement you have in
place.}

\wss{For most teams this section will just state that there is no confidential
information to protect.}
\section{IP to Protect}

\wss{State whether there is IP to protect.  If there is, point to the agreement.
All students who are working on a project that requires an IP agreement are also
required to sign the ``Intellectual Property Guide Acknowledgement.''}

\section{Copyright License}

\wss{What copyright license is your team adopting.  Point to the license in your
repo.}

\section{Team Meeting Plan}

\wss{How often will you meet? where?}

\wss{If the meeting is a physical location (not virtual), out of an abundance of
caution for safety reasons you shouldn't put the location online}

\wss{How often will you meet with your industry advisor?  when?  where?}

\wss{Will meetings be virtual?  At least some meetings should likely be
in-person.}

\wss{How will the meetings be structured?  There should be a chair for all meetings.  There should be an agenda for all meetings.}

\section{Team Communication Plan}

\wss{Issues on GitHub should be part of your communication plan.}

\section{Team Member Roles}

\wss{You should identify the types of roles you anticipate, like notetaker,
leader, meeting chair, reviewer.  Assigning specific people to those roles is
not necessary at this stage.  In a student team the role of the individuals will
likely change throughout the year.}

\section{Workflow Plan}

\begin{itemize}
	\item How will you be using git, including branches, pull request, etc.?
	\item How will you be managing issues, including template issues, issue
	classification, etc.?
  \item Use of CI/CD
\end{itemize}

\section{Project Decomposition and Scheduling}

\begin{itemize}
  \item Our team will use Github Projects as our core tool for task tracking and project management. Each feature will be decomposed into specific, verifiable small tasks, these tasks will be created in the form of Issues in Github Kanban Board. The Kanban board is divided into 4 stages: Todo, In Progress, Review and Done to visualize the workflow and ensure accountability. Team members will be assigned to different tasks, related pull request will be tied to these issues to maintain tracibility.
  \item Link to our Kanban project: \url{https://github.com/orgs/The-Crazy-Four-Games/projects}
  \item Decomposed Schedule:
	\begin{itemize}
		\item Week 03: Team Formed, Project Selected
			\begin{itemize}
			\item Create Github repo
			\item Assign intial role
			\item Decide on final project selection
			\end{itemize}		
		\item Week 04: Problem Statement, POC, Development Plan
			\begin{itemize}
			\item Draft initial problem statement and development plan
			\item Set up CI/CD pipeline and Github branch strategy
			\item Discuss the final programming languages and frameworks
			\end{itemize}
		\item Week 06: SRS + Hazard Analysis (Rev. 0)
			\begin{itemize}
			\item Draft System Requirements Specification (SRS)
			\item Identify hazards and mitigation strategy
			\end{itemize}
		\item Week 08: V\&V Plan (Rev. 0)
			\begin{itemize}
			\item Define V\&V strategy
			\item Break down into different tests and assigned responsibilities
			\end{itemize}
		\item Week 10: Design Document (Rev.-1)
			\begin{itemize}
			\item Decompose system into major modules (frontend, backend, db, api, etc.)
			\item Document architecture diagrams
			\end{itemize}
		\item Week 11-12: Proof of Concept Demonstration
			\begin{itemize}
			\item Implement minimum working prototype
			\item Prepare slides and live demo for demo presentation
			\end{itemize}
		\item Week 16: Design Document (Rev.0)
			\begin{itemize}
			\item Refine design based on the feedback of POC Demo
			\item Add details for extensibility and scalability
			\end{itemize}	
		\item Week 18-19: Revision 0  Demonstration
			\begin{itemize}
			\item Implement key features and workflows
			\item Conduct internal testing and bug fixing
			\item Prepare live demo
			\end{itemize}	
		\item Week 22: V\&V Report Revision 0
			\begin{itemize}
			\item Execute test plan and record results
			\item Analyze test coverage and tracibility
			\end{itemize}	
		\item Week 24: Final Demonstration (Rev.1)
			\begin{itemize}
			\item Finalize all core features
			\item Optimize performance
			\item Conduct mock presentation and user feedbacks
			\end{itemize}	
		\item Week 26: EXPO Demonstration
			\begin{itemize}
			\item Prepare polished project presentation for EXPO
			\end{itemize}	
		\item Week 26: Final Documentation (Rev.1)
			\begin{itemize}
			\item Finalize all documents
			\end{itemize}	
	\end{itemize}
\end{itemize}


\section{Proof of Concept Demonstration Plan}

Main Risk

\begin{itemize}
	\item Implementation Complexity: Correctly enforcing the gameplay mechanism (e.g. turn-taking,  rule variation, base-12)
	\item Ensuring required libraries and frameworks install and integrate smoothly across team members’ environments.
	
\end{itemize}
PoC Demonstration Goals \\In our PoC demonstration, we will address these risks by showing:
\begin{itemize}
	\item A working prototype of Crazy Eights where players can take turns, match cards by suit/rank, and play an "8" as a special 
	\item Successful integration between frontend and backend for real-time gameplay.
\end{itemize}

\section{Expected Technology}


\begin{itemize}
\item JavaScript, TypeScript
\item PostgreSQL
\item Github, git, Github projects
\item Node.js,  React
\item Jest
\end{itemize}


\section{Coding Standard}

\begin{itemize}
	\item Quality-Oriented Development: Code should be clear, maintainable and consistent across the whole project
	\item Requirement and Specification-Based: All implementation will be tied to the requirements and specifications
	\item Defensive Programming: Follow practices that reduce errors and improve robustness
\end{itemize}

\newpage{}

\section*{Appendix --- Reflection}

\wss{Not required for CAS 741}


\begin{enumerate}
    \item Why is it important to create a development plan prior to starting the
    project?\\ A clear development plan provides the team with structure and direction, it ensures that all team members have a common understanding of the project's goals, responsibilities and timelines. It can also reduce uncertainty, help the team identify risks early, and make it easier to track progress.
    \item In your opinion, what are the advantages and disadvantages of using
    CI/CD?\\ Advantages: CI/CD automates testing and deployment,  it reduces human errors and improves code quality.  Since it encourages frequent integration, problems are detected early. \\Disadvantages: Setting up CI/CD pipelines can be time-consuming, especially for small teams, it may introduce overhead if the project scope is small or team members are not familiar with this tool.
    \item What disagreements did your group have in this deliverable, if any,
    and how did you resolve them?\\One disagreement our group had was whether to focus solely on developing the Crazy Eights card game, or to make it into a product line that could support multiple card games/number systems,  some members felt that the product line idea would make the project impressive and ambitious, while others were concerned about the limited timeline and feasibility. After a team discussion and seeking advice from the professor and supervisor, we resolved this by agreeing to prioritize Crazy Eights as the core deliverable, ensuring we can deliver a complete and functional game. At the same time, we left the product line concept as a strech goal that could be pursued if time and resources permit. This compromise allowed us to balance ambition with practicality, while keeping the team aligned.
\end{enumerate}

\newpage{}

\section*{Appendix --- Team Charter}

\wss{borrows from
\href{https://engineering.up.edu/industry_partnerships/files/team-charter.pdf}
{University of Portland Team Charter}}

\subsection*{External Goals}

\begin{itemize}
	\item Deliver a polished product at the EXPO with hope of receiving positive feedback
	\item Build a project that can be showcased in future interviews and portfolios
	\item Aim for a strong course grade by following best practices and meeting all deliverable expectations
	\item Strengthen our knowlegde of modern frameworks so that the project also contributes to our long-term career growth.
\end{itemize}

\subsection*{Attendance}

\subsubsection*{Expectations}

Team members are expected to attend scheduled meetings(in-person or via Discord) on time and stay until the meeting is concluded.  Consistent attendance is essential to maintain good communication and progress.

\subsubsection*{Acceptable Excuse}

Acceptable excuses for missing a meeting or a deadline include illness, family emergencies, or unavoidable academic conflicts.Unacceptable excuses include forgetting, oversleeping, etc.

\subsubsection*{In Case of Emergency}

If a team member experiences an emergency, they should notify the team as soon as possible through Discord. They should also provide updates on the status of their assigned tasks and, if neccessary, delegate or share their work so that the team can adjust and continue meeting deadlines.

\subsection*{Accountability and Teamwork}

\subsubsection*{Quality} 

\wss{What are your team's expectations regarding the quality
of team members' preparation for team meetings and the quality of the
deliverables that members bring to the team?}

\subsubsection*{Attitude}

\wss{What are your team's expectations regarding team members' ideas,
interactions with the team, cooperation, attitudes, and anything else regarding
team member contributions?  Do you want to introduce a code of conduct?  Do you
want a conflict resolution plan?  Can adopt existing codes of conduct.}

\subsubsection*{Stay on Track}

\wss{What methods will be used to keep the team on track? How will your team
ensure that members contribute as expected to the team and that the team
performs as expected? How will your team reward members who do well and manage
members whose performance is below expectations?  What are the consequences for
someone not contributing their fair share?}

\wss{You may wish to use the project management metrics collected for the TA and
instructor for this.}

\wss{You can set target metrics for attendance, commits, etc.  What are the
consequences if someone doesn't hit their targets?  Do they need to bring the
coffee to the next team meeting?  Does the team need to make an appointment with
their TA, or the instructor?  Are there incentives for reaching targets early?}

\subsubsection*{Team Building}

\wss{How will you build team cohesion (fun time, group rituals, etc.)? }

\subsubsection*{Decision Making} 

\wss{How will you make decisions in your group? Consensus?  Vote? How will you
handle disagreements? }

\end{document}