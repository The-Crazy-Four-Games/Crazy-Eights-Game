\documentclass{article}

\usepackage{booktabs}
\usepackage{tabularx}

\title{Development Plan\\The Crazy Eights}

\author{Ruida Chen \\ Alvin Qian \\ Ammar, Sharbat \\ Jiaming Li}


\date{September 2025}

%% Comments

\usepackage{color}

\newif\ifcomments\commentstrue %displays comments
%\newif\ifcomments\commentsfalse %so that comments do not display

\ifcomments
\newcommand{\authornote}[3]{\textcolor{#1}{[#3 ---#2]}}
\newcommand{\todo}[1]{\textcolor{red}{[TODO: #1]}}
\else
\newcommand{\authornote}[3]{}
\newcommand{\todo}[1]{}
\fi

\newcommand{\wss}[1]{\authornote{magenta}{SS}{#1}} 
\newcommand{\plt}[1]{\authornote{cyan}{TPLT}{#1}} %For explanation of the template
\newcommand{\an}[1]{\authornote{cyan}{Author}{#1}}

%% Common Parts

\newcommand{\progname}{ProgName} % PUT YOUR PROGRAM NAME HERE
\newcommand{\authname}{Team \#, Team Name
\\ Student 1 name
\\ Student 2 name
\\ Student 3 name
\\ Student 4 name} % AUTHOR NAMES                  

\usepackage{hyperref}
    \hypersetup{colorlinks=true, linkcolor=blue, citecolor=blue, filecolor=blue,
                urlcolor=blue, unicode=false}
    \urlstyle{same}
                                


\begin{document}

\maketitle

\begin{table}[hp]
\caption{Revision History} \label{TblRevisionHistory}
\begin{tabularx}{\textwidth}{llX}
\toprule
\textbf{Date} & \textbf{Developer(s)} & \textbf{Change}\\
\midrule
Sep 21 & Ruida Chen & Worked on section 8-12 of dev plan and first half of appendix\\
Sep 22 & Alvin Qian & Worked on section 1-7 of dev plan and second half of appendix\\
... & ... & ...\\
\bottomrule
\end{tabularx}
\end{table}

\newpage{}

\section{Introduction}

This Development Plan defines how the Crazy Eights project will be organized and executed. It establishes scope, objectives, team structure, communication methods, workflow conventions, schedule, risk mitigation focus, technology stack, and coding standards.

\section{Confidential Information}

There is no confidential or proprietary (industry) information in this project. No NDAs or confidentiality agreements are in place.

\section{IP to Protect}

There is no external partner IP and the team does not plan to pursue patents. All work is collectively authored by the team and will be released under the selected open source license.

\section{Copyright License}

The project is released under the MIT License. All source code and documents may be used, modified, and redistributed in accordance with that license.

\section{Team Meeting Plan}

\subsection*{Meeting Frequency}
\begin{itemize}
  \item Regular weekly meetings:
  \begin{itemize}
    \item One in-person meeting during scheduled tutorial time
    \item One online meeting via Discord (time TBD by team availability)
  \end{itemize}
  \item Additional ad hoc meetings scheduled as needed for deadlines or unblocking issues
  \item Bi-weekly supervisor meetings with Dr.~Paul Rapoport (virtual or in-person)
\end{itemize}

\subsection*{Meeting Format}
\begin{itemize}
  \item Primary format: Virtual meetings via Discord (voice channel)
  \item Secondary format: In-person meetings during class/tutorials
  \item Meeting roles (rotating):
  \begin{itemize}
    \item \textbf{Chair}: Responsible for
    \begin{itemize}
      \item Preparing and sharing agenda (12+ hours in advance)
      \item Managing time allocation
      \item Facilitating inclusive discussion
    \end{itemize}
  \item \textbf{Note-taker}: Records
    \begin{itemize}
      \item Key decisions
      \item Action items with owners and deadlines
      \item Risks and open questions
    \end{itemize}
  \end{itemize}
\end{itemize}

\subsection*{Attendance Policy}
If unable to attend:
\begin{itemize}
  \item Notify team via Discord in advance
  \item Review meeting notes independently
  \item Complete assigned tasks asynchronously
\end{itemize}

\section{Team Communication Plan}
We will use multiple coordinated channels to ensure clear, persistent, and efficient communication:

\begin{itemize}
  \item \textbf{GitHub}: Version control, issue tracking, pull request review, and Kanban board management. 
  \item \textbf{Discord}: Primary platform for quick text and voice communication. Daily informal updates, coordination of short-term tasks, and quick problem-solving occur here. Urgent blockers are first raised in Discord.
  \item \textbf{In-Person (Tutorial / Scheduled Check-ins)}: Used for structured milestone work, live demonstrations, and supervisor touchpoints.
\end{itemize}

\section{Team Member Roles}
Roles are rotational to ensure balanced workload distribution, broaden experience, and encourage shared ownership. Anticipated roles include:

\begin{itemize}
  \item \textbf{Team Liaison (Ammar Sharbat)}: Primary point of contact with the supervisor and course staff; coordinates meeting requests and relays external feedback.
  \item \textbf{Developer (All members)}: Implements assigned features, writes and maintains tests, and updates related documentation.
  \item \textbf{Reviewer (All members)}: Performs code reviews for pull requests, checks alignment with coding standards, and leaves comments if needed before approval.
  \item \textbf{Meeting Chair/Note-taker (All members rotate)}: Facilitates meetings, ensures agenda is followed, and documents key discussion points and action items.
\end{itemize}


\section{Workflow Plan}

\begin{enumerate}
  \item \textbf{Update Local Repository}: Pull the latest changes from the `main` branch to ensure the local repository is up to date.
  \item \textbf{Branching}: Create a new feature (or fix) branch from `main`. Follow a consistent naming convention such as: feature/short-description or fix/issue-id.
  \item \textbf{Coding}: Implement modules and functions according to the design and requirements. Follow the agreed coding standards and document code with comments where necessary.
  \item \textbf{Unit Testing}: Write and execute unit tests for the newly implemented modules/functions to check expected behaviour. Tests must pass locally before pushing to the branch.
  \item \textbf{Commit and Push}: Commit changes with descriptive messages. Push to the remote feature branch.
  \item \textbf{Review and Merge}: Create a pull request into `main`. A team member reviews for feature logic, style, test coverage, and documentation updates. The reviewer will leave comments and suggestions if needed. After at least one approval, the branch is merged into the main branch.
\end{enumerate}

All project tasks are tracked through GitHub Issues and a Kanban board using GitHub Projects (Backlog / In Progress / Review / Done).

\section{Project Decomposition and Scheduling}

\begin{itemize}
  \item Our team will use Github Projects as our core tool for task tracking and project management. Each feature will be decomposed into specific, verifiable small tasks, these tasks will be created in the form of Issues in Github Kanban Board. The Kanban board is divided into 4 stages: Todo, In Progress, Review and Done to visualize the workflow and ensure accountability. Team members will be assigned to different tasks, related pull request will be tied to these issues to maintain traceability.
  \item Link to our Kanban project: \url{https://github.com/orgs/The-Crazy-Four-Games/projects}
  \item Decomposed Schedule:
	\begin{itemize}
		\item Week 03: Team Formed, Project Selected
			\begin{itemize}
			\item Create Github repo
			\item Assign initial role
			\item Decide on final project selection
			\end{itemize}		
		\item Week 04: Problem Statement, POC, Development Plan
			\begin{itemize}
			\item Draft initial problem statement and development plan
			\item Set up CI/CD pipeline and Github branch strategy
			\item Discuss the final programming languages and frameworks
			\end{itemize}
		\item Week 06: SRS + Hazard Analysis (Rev. 0)
			\begin{itemize}
			\item Draft System Requirements Specification (SRS)
			\item Identify hazards and mitigation strategy
			\end{itemize}
		\item Week 08: V\&V Plan (Rev. 0)
			\begin{itemize}
			\item Define V\&V strategy
			\item Break down into different tests and assigned responsibilities
			\end{itemize}
		\item Week 10: Design Document (Rev.-1)
			\begin{itemize}
			\item Decompose system into major modules (frontend, backend, db, api, etc.)
			\item Document architecture diagrams
			\end{itemize}
		\item Week 11-12: Proof of Concept Demonstration
			\begin{itemize}
			\item Implement minimum working prototype
			\item Prepare slides and live demo for demo presentation
			\end{itemize}
		\item Week 16: Design Document (Rev.0)
			\begin{itemize}
			\item Refine design based on the feedback of POC Demo
			\item Add details for extensibility and scalability
			\end{itemize}	
		\item Week 18-19: Revision 0  Demonstration
			\begin{itemize}
			\item Implement key features and workflows
			\item Conduct internal testing and bug fixing
			\item Prepare live demo
			\end{itemize}	
		\item Week 22: V\&V Report Revision 0
			\begin{itemize}
			\item Execute test plan and record results
			\item Analyze test coverage and traceability
			\end{itemize}	
		\item Week 24: Final Demonstration (Rev.1)
			\begin{itemize}
			\item Finalize all core features
			\item Optimize performance
			\item Conduct mock presentation and user feedbacks
			\end{itemize}	
		\item Week 26: EXPO Demonstration
			\begin{itemize}
			\item Prepare polished project presentation for EXPO
			\end{itemize}	
		\item Week 26: Final Documentation (Rev.1)
			\begin{itemize}
			\item Finalize all documents
			\end{itemize}	
	\end{itemize}
\end{itemize}


\section{Proof of Concept Demonstration Plan}

Main Risk

\begin{itemize}
	\item Implementation Complexity: Correctly enforcing the gameplay mechanism (e.g. turn-taking,  rule variation, base-12)
	\item Ensuring required libraries and frameworks install and integrate smoothly across team members’ environments.
	
\end{itemize}
PoC Demonstration Goals \\In our PoC demonstration, we will address these risks by showing:
\begin{itemize}
	\item A working prototype of Crazy Eights where players can take turns, match cards by suit/rank, and play an "8" as a special 
	\item Successful integration between frontend and backend for real-time gameplay.
\end{itemize}

\section{Expected Technology}


\begin{itemize}
\item JavaScript, TypeScript
\item PostgreSQL
\item Github, git, Github projects
\item Node.js,  React
\item Jest
\end{itemize}


\section{Coding Standard}

\begin{itemize}
	\item Quality-Oriented Development: Code should be clear, maintainable and consistent across the whole project
	\item Requirement and Specification-Based: All implementation will be tied to the requirements and specifications
	\item Defensive Programming: Follow practices that reduce errors and improve robustness
\end{itemize}

\newpage{}

\section*{Appendix --- Reflection}


\begin{enumerate}
    \item Why is it important to create a development plan prior to starting the
    project?\\ A clear development plan provides the team with structure and direction, it ensures that all team members have a common understanding of the project's goals, responsibilities and timelines. It can also reduce uncertainty, help the team identify risks early, and make it easier to track progress.
    \item In your opinion, what are the advantages and disadvantages of using
    CI/CD?\\ Advantages: CI/CD automates testing and deployment,  it reduces human errors and improves code quality.  Since it encourages frequent integration, problems are detected early. \\Disadvantages: Setting up CI/CD pipelines can be time-consuming, especially for small teams, it may introduce overhead if the project scope is small or team members are not familiar with this tool.
    \item What disagreements did your group have in this deliverable, if any,
    and how did you resolve them?\\One disagreement our group had was whether to focus solely on developing the Crazy Eights card game, or to make it into a product line that could support multiple card games/number systems,  some members felt that the product line idea would make the project impressive and ambitious, while others were concerned about the limited timeline and feasibility. After a team discussion and seeking advice from the professor and supervisor, we resolved this by agreeing to prioritize Crazy Eights as the core deliverable, ensuring we can deliver a complete and functional game. At the same time, we left the product line concept as a stretch goal that could be pursued if time and resources permit. This compromise allowed us to balance ambition with practicality, while keeping the team aligned.
\end{enumerate}

\newpage{}

\section*{Appendix --- Team Charter}


\subsection*{External Goals}

\begin{itemize}
	\item Deliver a polished product at the EXPO with hope of receiving positive feedback
	\item Build a project that can be showcased in future interviews and portfolios
	\item Aim for a strong course grade by following best practices and meeting all deliverable expectations
	\item Strengthen our knowledge of modern frameworks so that the project also contributes to our long-term career growth.
\end{itemize}

\subsection*{Attendance}

\subsubsection*{Expectations}

Team members are expected to attend scheduled meetings(in-person or via Discord) on time and stay until the meeting is concluded.  Consistent attendance is essential to maintain good communication and progress.

\subsubsection*{Acceptable Excuse}

Acceptable excuses for missing a meeting or a deadline include illness, family emergencies, or unavoidable academic conflicts.Unacceptable excuses include forgetting, oversleeping, etc.

\subsubsection*{In Case of Emergency}

If a team member experiences an emergency, they should notify the team as soon as possible through Discord. They should also provide updates on the status of their assigned tasks and, if necessary, delegate or share their work so that the team can adjust and continue meeting deadlines.

\subsection*{Accountability and Teamwork}

\subsubsection*{Quality} 

\begin{itemize}
  \item \textbf{Meeting Preparation:}
  \begin{itemize}
    \item Read the agenda and relevant issues/PRs prior to the meeting.
    \item Come prepared with (a) a concise progress update, (b) explicit blockers, (c) questions to ask.
  \item For design/architecture discussions, read any documentation shared at least 12 hours before the meeting.
  \end{itemize}
  \item \textbf{Expectation for a code task / issue:}
  \begin{itemize}
  \item Code compiles/runs locally without new warnings or linter errors.
  \item Automated tests added or updated: cover new logic and edge cases.
    \item All tests pass locally.
  \item Documentation updated: inline comments for non-obvious logic; README/module documentation updated if behaviour, interfaces, or run steps change.
    \item Peer review completed (minimum one approval) with review comments addressed or explicitly deferred via a follow-up issue.
  \end{itemize}
  \item \textbf{Timeliness:} Work items should be completed within their originally estimated iteration window.
\end{itemize}

\subsubsection*{Attitude}

\begin{itemize}
  \item \textbf{Respect and Inclusion:} Listen actively and avoid interrupting. Credit ideas to originators; disagreements focus on the idea, never the person.
  \item \textbf{Communication Responsiveness:} Weekday Discord messages acknowledged within 24 hours (a reaction or short reply). If unavailable (midterms, travel), communicate ahead of time if possible.
  \item \textbf{Constructive Feedback:} Use specific, friendly, actionable language and pair it with reasoning.
  \item \textbf{Conflict of Ideas:} Healthy debate is encouraged, and once a decision is recorded, team members support it unless new evidence emerges.
  \item \textbf{Inclusivity:} Zero tolerance for harassment, discrimination, or disparaging language. Maintain a commitment to inclusivity across backgrounds and experience levels.
\end{itemize}

\paragraph{Conflict Resolution Process}
\begin{enumerate}
  \item \textit{Direct Discussion:} Involved members attempt a private, respectful conversation to clarify intent and desired outcome.
  \item \textit{Mediated Conversation:} If unresolved, bring the issue to the next meeting or request a teammate to facilitate a short discussion.
  \item \textit{Escalation:} If behaviour breaches the Code of Conduct, escalate to the supervisor / TA. Persistent issues may be elevated to the instructor.
\end{enumerate}

\subsubsection*{Stay on Track}
The team will use clear metrics and regular check-ins to ensure progress and accountability.

\paragraph{Metrics Tracked}
\begin{itemize}
  \item Meeting attendance (tutorials, scheduled meetings, supervisor check-ins)
  \item Number of pull requests made and contributions to reviewing PRs
\end{itemize}

\paragraph{Recognition}
Positive, on-time, high-quality contributions are acknowledged verbally in weekly meetings. Reusable good practices or discoveries are shared for team-wide adoption.

\paragraph{Managing Underperformance}
\begin{enumerate}
  \item Check-in to clarify blockers (scope, skill gap, time constraints).
  \item Adjust: refine issue scope, pair program, or provide targeted resources.
  \item If no improvement and no proactive communication: document an action plan with achievable concrete milestones.
  \item Continued gaps will need to be escalated to TA / instructor for guidance.
\end{enumerate}

\paragraph{Consequences}
If deadlines are repeatedly missed without notice:
\begin{itemize}
  \item Action plan logged (issue comment or team log) with dates.
  \item Escalation path: (1) team check-in, (2) written plan, (3) supervisor/TA, (4) instructor if unresolved.
\end{itemize}

\subsubsection*{Team Building}
Lightweight cohesion practices:
\begin{itemize}
  \item Open casual discussion on any topic for the first 5 minutes of each meeting.
  \item Celebrate small wins (merged PRs, resolved issues) in meetings.
  \item Share interesting articles, tools, or tips related to the project in Discord.
\end{itemize}


\subsubsection*{Decision Making}
\begin{enumerate}
  \item Open discussion aiming for consensus.
  \item If no consensus, move on to a simple majority vote.
  \item Record decision (issue comment, PR, or decision log); set as the final decision until new evidence justifies a change.
\end{enumerate}

\end{document}