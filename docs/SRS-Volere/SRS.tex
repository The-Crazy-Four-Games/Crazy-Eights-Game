% THIS DOCUMENT IS FOLLOWS THE VOLERE TEMPLATE BY Suzanne Robertson and James Robertson
% ONLY THE SECTION HEADINGS ARE PROVIDED
%
% Initial draft from https://github.com/Dieblich/volere
%
% Risks are removed because they are covered by the Hazard Analysis
\documentclass[12pt]{article}

\usepackage{booktabs}
\usepackage{tabularx}
\usepackage{hyperref}
\hypersetup{
    bookmarks=true,         % show bookmarks bar?
      colorlinks=true,      % false: boxed links; true: colored links
    linkcolor=red,          % color of internal links (change box color with linkbordercolor)
    citecolor=green,        % color of links to bibliography
    filecolor=magenta,      % color of file links
    urlcolor=cyan           % color of external links
}

\newcommand{\lips}{\textit{Insert your content here.}}

\input{../Comments}
\input{../Common}

\begin{document}

\title{Software Requirements Specification for \progname: subtitle describing software} 
\author{\authname}
\date{\today}
    
\maketitle

~\newpage

\pagenumbering{roman}

\tableofcontents

~\newpage

\section*{Revision History}

\begin{tabularx}{\textwidth}{p{3cm}p{2cm}X}
\toprule {\textbf{Date}} & {\textbf{Version}} & {\textbf{Notes}}\\
\midrule
Date 1 & 1.0 & Notes\\
Date 2 & 1.1 & Notes\\
\bottomrule
\end{tabularx}

~\\

~\newpage
\section{Purpose of the Project}
\subsection{User Business}
The purpose of this project is to design and implement an educational card game based on the traditional \textit{Crazy 8s} rule set, but adapted to integrate the \textbf{Dozenal (base-12) number system}.
\begin{itemize}
    \item This project addresses the lack of accessible and engaging tools that introduce alternative number systems in a playful and intuitive way.
    \item By combining a familiar card game mechanic with Dozenal representations and operations, users can gradually build comfort and intuition with the base-12 system.
    \item The primary business value lies in providing a lightweight, fun, and interactive educational tool for students, hobbyists, and anyone interested in number systems beyond decimal.
\end{itemize}
Additionally, the game offers an opportunity to evaluate how gamification can support mathematical learning, and whether abstract concepts (such as base conversions or divisibility in Dozenal) can be effectively taught through play.

\subsection{Goals of the Project}
The goals of this project are:
\begin{itemize}
    \item Educational Integration: Seamlessly incorporate Dozenal concepts (symbols 0--B, factorization, arithmetic) into the gameplay, ensuring that players learn by playing without requiring formal prior knowledge.
    \item Gameplay Design: Deliver a working digital version of \textit{Crazy 8s} that is intuitive, responsive, and enjoyable, while maintaining the familiar flow of the original game and introducing Dozenal-specific mechanics (e.g., matching rules, scoring, or special cards).
    \item Accessibility and Engagement: Create a user-friendly interface that lowers the barrier to learning, accessible for casual users while offering depth for learners who want to explore Dozenal further.
    \item Scalability / Stretch Goals: Explore the potential for extending the system to other educational card or board games, and investigate how different number bases can be taught through similar game mechanics.
\end{itemize}
\section{Stakeholders}
\subsection{Client}
\lips
\subsection{Customer}
\lips
\subsection{Other Stakeholders}
\lips
\subsection{Hands-On Users of the Project}
\lips
\subsection{Personas}
\lips
\subsection{Priorities Assigned to Users}
\lips
\subsection{User Participation}
\lips
\subsection{Maintenance Users and Service Technicians}
\lips

\section{Mandated Constraints}
\subsection{Solution Constraints}
\lips
\subsection{Implementation Environment of the Current System}
\lips
\subsection{Partner or Collaborative Applications}
\lips
\subsection{Off-the-Shelf Software}
\lips
\subsection{Anticipated Workplace Environment}
\lips
\subsection{Schedule Constraints}
\lips
\subsection{Budget Constraints}
\lips
\subsection{Enterprise Constraints}
\lips

\section{Naming Conventions and Terminology}
\subsection{Glossary of All Terms, Including Acronyms, Used by Stakeholders
involved in the Project}

\begin{itemize}
    \item \textbf{Base-10 (Decimal)}: The standard numerical system using ten digits (0--9), commonly used in arithmetic and everyday calculations.
    \item \textbf{Base-12 (Dozenal)}: A numerical system using twelve digits (0--9, A, B), where A represents 10 and B represents 11 in decimal. Has divisibility advantages (divisors: 2, 3, 4, 6).
    \item \textbf{Crazy Eights}: A classic card game where players match cards by suit or rank, with the ``8'' card acting as a wild card allowing the player to declare a new suit.
    \item \textbf{MVP}: Minimum Viable Product, the initial version of the Crazy Eights software with core functionality, including two-player gameplay, classic rules, and dozenal scoring.
    \item \textbf{Functional Goals}: Features and behaviors the software must implement, such as gameplay mechanics and dozenal score display.
    \item \textbf{Non-functional Goals}: Quality attributes of the software, such as usability, performance, and stability.
    \item \textbf{Stretch Goals}: Optional features or enhancements, such as multiplayer support or advanced dozenal rule variants, to be implemented if time permits.
    \item \textbf{GitHub}: The platform used for version control, issue tracking, and Kanban board management.
    \item \textbf{Discord}: The communication platform for team coordination, quick updates, and voice meetings.
    \item \textbf{Kanban Board}: A project management tool in GitHub Projects, divided into stages (Backlog, In Progress, Review, Done) to track tasks.
    \item \textbf{CI/CD}: Continuous Integration/Continuous Deployment, an automated process for testing and deploying code changes.
    \item \textbf{SRS}: Software Requirements Specification, this document outlining the requirements for the Crazy Eights project.
    \item \textbf{PoC}: Proof of Concept, a prototype demonstrating core gameplay mechanics to validate feasibility.
    \item \textbf{UI}: User Interface, the visual and interactive components of the software, such as the game board and score display.
\end{itemize}

\section{Relevant Facts And Assumptions}
\subsection{Relevant Facts}
\lips
\subsection{Business Rules}
\lips
\subsection{Assumptions}
\lips

\section{The Scope of the Work}
\subsection{The Current Situation}

The current numerical system used around the world is predominantly decimal (base-10). This system, while widely adopted, has limitations in representing fractions cleanly, as its prime factors (2 and 5) result in recurring decimals for simple ratios 1/3 or 1/6. On the other hand, the dozenal (base-12) system, with divisors 2, 3, 4, and 6, offers a more intuitive and concise fraction representations, historically used in trade and measurement systems (e.g., dozens, hours). However, dozenal is underutilized in education and practice, leaving students, educators, and professionals reliant on decimal despite its inefficiencies for certain calculations. There is no current fun and engaging way to demonstrate the practical advantages of dozenal, such as a card game like Crazy Eights, to promote its adoption.

\subsection{The Context of the Work}

The project focuses on the software implementation of the Crazy Eights card game that incorporates dozenal (base-12) scoring and display to highlight the benefits of the dozenal system. The context includes:

\begin{itemize}
    \item \textbf{Educational Context}: The project will simplify mathematical understanding for the user by showing dozenal's advantages in a familiar and fun game format.
    \item \textbf{Technical Context}: The software will be developed using a modern tech stack for web applications (JavaScript, TypeScript, Node.js, React, PostgreSQL) and use GitHub CI/CD pipelines for version control and testing.
    \item \textbf{Stakeholder Context}: Key stakeholders include students, educators, mathematicians, computer scientists, and the general public, all of whom could benefit from clearer fraction representations and easier mental arithmetic.
\end{itemize}

\subsection{Work Partitioning}

The workload is divided into the following major phases, aligned with the development plan:

\begin{enumerate}
    \item \textbf{Problem Definition and Planning (Weeks 3--4)}: Draft problem statement, development plan, and initial proof of concept (PoC) to establish scope and feasibility.
    \item \textbf{Requirements and Hazard Analysis (Week 6)}: Develop the SRS and identify potential risks and mitigation strategies.
    \item \textbf{Verification and Validation Planning (Week 8)}: Define testing strategies to ensure the software meets functional and non-functional requirements.
    \item \textbf{System Design (Weeks 10--16)}: Create and refine architecture diagrams, decompose the system into modules (frontend, backend, database, API), and document extensibility.
    \item \textbf{Implementation and Testing (Weeks 11--19)}: Develop the MVP (two-player game with classic rules and dozenal scoring), conduct unit and integration testing, and prepare for demonstrations.
    \item \textbf{Demonstrations and Refinement (Weeks 19--24)}: Conduct Revision 0 and final demonstrations, incorporating feedback to improve functionality and performance.
    \item \textbf{Final Documentation and EXPO (Week 26)}: Finalize documentation and present a polished product at the EXPO.
\end{enumerate}

Each task is tracked via GitHub Issues and the Kanban board, with responsibilities assigned to team members based on rotating roles (developer, reviewer, meeting chair, note-taker).

\subsection{Specifying a Business Use Case (BUC)}

\textbf{BUC: Play a Game of Crazy Eights with Dozenal Scoring}

\begin{itemize}
    \item \textbf{Actors}: Two players, the software system.
    \item \textbf{Trigger}: A player creates a new game session using the UI.
    \item \textbf{Description}: Two players take turns matching cards by suit or rank, with the ``8'' card acting as a wild card that allows the player to declare a new suit. If no valid move can be made, the player draws from the stock pile. The game ends when one player discards all their cards, and the score is calculated and displayed in dozenal notation.
    \item \textbf{Preconditions}: The user is logged in. The software is running on a web interface, with a functional UI displaying the hand, discard pile, stock pile, and score tracker.
    \item \textbf{Postconditions}: The game concludes with a winner, points are displayed in dozenal and must be counted by the users to see the final score. The user can log Off or start a new game.
    \item \textbf{Basic Flow}:
    \begin{enumerate}
        \item The system deals cards to both players and initializes the discard pile with a starter card.
        \item Players take turns, selecting a card to play (matching suit or rank) or drawing from the stock pile.
        \item If an ``8'' is played, the player selects a new suit via the UI.
        \item The system checks moves and gives immediate feedback for invalid moves.
        \item The game keeps going until one player has no cards left, triggering user score counting and calculation challenge in dozenal.
        \item The system displays the final score.
    \end{enumerate}
    \item \textbf{Exceptions}:
    \begin{itemize}
        \item Invalid move attempted: The system highlights the error and prompts the player to select a valid card or draw.
        \item Stock pile exhausted: The system reshuffles the discard pile to replenish the stock pile.
    \end{itemize}
    \item \textbf{Assumptions}: Players are familiar with basic card game mechanics and the system supports a stable internet connection for play.
\end{itemize}

\section{Business Data Model and Data Dictionary}
\subsection{Business Data Model}

These are the key entities and relationships involved in the Crazy Eights game with dozenal scoring.

\begin{itemize}
    \item \textbf{Entities}:
    \begin{itemize}
        \item \textbf{Player}: Represents a game participant, holding a hand of cards and a score.
        \item \textbf{Card}: Represents a single playing card with a suit and rank.
        \item \textbf{Deck}: A collection of cards, divided into the stock pile and discard pile.
        \item \textbf{Game Session}: Tracks the state of a single game, including players, current turn, discard pile, and scores.
        \item \textbf{Score}: Tracks points in dozenal notation, calculated based on game rules.
    \end{itemize}
    \item \textbf{Relationships}:
    \begin{itemize}
        \item A \textbf{Game Session} has 2 \textbf{Players} (MVP) or 2--4 \textbf{Players} (stretch goal).
        \item Each \textbf{Player} has a \textbf{Hand} (subset of \textbf{Cards}).
        \item A \textbf{Deck} consists of 52 \textbf{Cards}, split into \textbf{Stock Pile} and \textbf{Discard Pile}.
        \item A \textbf{Game Session} produces a \textbf{Score} for each \textbf{Player} in dozenal notation.
        \item A \textbf{Card} played in a \textbf{Game Session} affects the \textbf{Discard Pile} and may trigger a suit change (if an ``8'').
    \end{itemize}
\end{itemize}

\subsection{Data Dictionary}

\begin{itemize}
    \item \textbf{Player}:
    \begin{itemize}
        \item \textbf{ID}: Unique identifier for a player (integer).
        \item \textbf{Name}: Display name for the player (string).
        \item \textbf{Hand}: List of cards held by the player (array of Card objects).
        \item \textbf{Score}: Player's cumulative score in dozenal notation (string, such as ``1A'' for 22 in decimal).
    \end{itemize}
    \item \textbf{Card}:
    \begin{itemize}
        \item \textbf{Suit}: One of four suits (Hearts, Diamonds, Clubs, Spades) (string).
        \item \textbf{Rank}: Card value (2--10, J, Q, K, A, 8) (string).
        \item \textbf{IsWild}: Boolean indicating if the card is an ``8'' (true/false).
    \end{itemize}
    \item \textbf{Deck}:
    \begin{itemize}
        \item \textbf{Stock Pile}: List of cards available for drawing (array of Card objects).
        \item \textbf{Discard Pile}: List of played cards (array of Card objects).
    \end{itemize}
    \item \textbf{Game Session}:
    \begin{itemize}
        \item \textbf{Session ID}: Unique identifier for the game session (string).
        \item \textbf{Current Turn}: ID of the player whose turn it is (integer).
        \item \textbf{Status}: Game state (Active, Completed) (string).
        \item \textbf{Starter Card}: The first card in the discard pile (Card object).
        \item \textbf{Last Played Suit}: The active suit after an ``8'' is played (string).
    \end{itemize}
    \item \textbf{Score}:
    \begin{itemize}
        \item \textbf{Player ID}: Links to the player (integer).
        \item \textbf{Dozenal Value}: Score in base-12 notation (string, such as ``14'' for 16 in decimal).
        \item \textbf{Calculation Method}: Rules for scoring (sum of remaining cards in the opponents hands) (string).
    \end{itemize}
\end{itemize}

\section{The Scope of the Product}
\subsection{Product Boundary}
\lips
\subsection{Product Use Case Table}
\lips
\subsection{Individual Product Use Cases (PUC's)}
\lips

\section{Functional Requirements}
\subsection{Functional Requirements}
\lips

\section{Look and Feel Requirements}
\subsection{Appearance Requirements}
\lips
\subsection{Style Requirements}
\lips

\section{Usability and Humanity Requirements}
\subsection{Ease of Use Requirements}
\lips
\subsection{Personalization and Internationalization Requirements}
\lips
\subsection{Learning Requirements}
\lips
\subsection{Understandability and Politeness Requirements}
\lips
\subsection{Accessibility Requirements}
\lips

\section{Performance Requirements}
\subsection{Speed and Latency Requirements}
\lips
\subsection{Safety-Critical Requirements}
\lips
\subsection{Precision or Accuracy Requirements}
\lips
\subsection{Robustness or Fault-Tolerance Requirements}
\lips
\subsection{Capacity Requirements}
\lips
\subsection{Scalability or Extensibility Requirements}
\lips
\subsection{Longevity Requirements}
\lips

\section{Operational and Environmental Requirements}
\subsection{Expected Physical Environment}
\lips
\subsection{Wider Environment Requirements}
\lips
\subsection{Requirements for Interfacing with Adjacent Systems}
\lips
\subsection{Productization Requirements}
\lips
\subsection{Release Requirements}
\lips

\section{Maintainability and Support Requirements}
\subsection{Maintenance Requirements}

\begin{itemize}
    \item \textbf{Code Maintainability}: The codebase must be modular, with clear separation of concerns to facilitate updates and debugging. Inline comments for non-obvious logic and updated README/module documentation are required, as specified in the development plan.
    \item \textbf{Version Control}: All changes must be tracked via GitHub, with descriptive commit messages and pull requests requiring at least one peer review to ensure maintainability.
    \item \textbf{Extensibility}: The system must support adding new features without requiring significant refactoring. This is achieved by using a modular architecture and Object oriented Design principles.
    \item \textbf{Documentation Updates}: Any change in behavior, interfaces, or setup instructions must be reflected in the documentation to ensure future developers can maintain the system. Local deployment instructions documented in the README to simplify setup for developers and maintainers.
\end{itemize}

\subsection{Supportability Requirements}

\begin{itemize}
    \item \textbf{Error Reporting}: The system must provide clear, user-friendly feedback for invalid moves. Logs or replays of game sessions must be available for debugging, as specified in the non-functional goals.
    \item \textbf{User Support}: A tutorial or visual guidance (stretch goal) must be provided to assist new players in understanding gameplay and dozenal scoring. This reduces the need for extensive manual support.
    \item \textbf{Cross-Platform Support}: The software must run on different web environments (Windows, macOS, or browsers). 
\end{itemize}

\subsection{Adaptability Requirements}

\begin{itemize}
    \item \textbf{Rule Flexibility}: The system must support toggling house rules (e.g., draw-until-playable, stacking eights) via a rule configurator (stretch goal), allowing adaptation to different play styles without code changes.
    \item \textbf{Scalability for Players}: The architecture must accommodate extending from two-player (MVP) to 3--4 player games (stretch goal) by modifying session management and turn logic.
    \item \textbf{Numeric System Extensibility}: The scoring system must allow switching between dozenal and decimal displays, ensuring adaptability for users unfamiliar with base-12.
\end{itemize}

\section{Security Requirements}
\subsection{Access Requirements}
\lips
\subsection{Integrity Requirements}
\lips
\subsection{Privacy Requirements}
\lips
\subsection{Audit Requirements}
\lips
\subsection{Immunity Requirements}
\lips

\section{Cultural Requirements}
\subsection{Cultural Requirements}

\begin{itemize}
    \item \textbf{Numeric System Accessibility}: The software must present dozenal (base-12) in an intuitive, non-disruptive way. This includes clear UI elements for dozenal scores (using A and B for 10 and 11) and optional tutorials (stretch goal) to explain dozenal notation to users unfamiliar with it.
    \item \textbf{Inclusivity}: The system must avoid cultural biases in its design, ensuring that gameplay and terminology (suits, ranks) are universally recognizable across cultures familiar with standard playing cards. No culturally specific references or imagery should be used in the UI to maintain broad accessibility.
    \item \textbf{Educational Alignment}: The software must align with educational goals by demonstrating the practical benefits of dozenal in a game context, making it appealing to students and educators. This supports the cultural shift toward exploring alternative numeric systems, as advocated by stakeholders like dozenal enthusiasts.
\end{itemize}

\section{Compliance Requirements}
\subsection{Legal Requirements}
\lips
\subsection{Standards Compliance Requirements}
\lips

\section{Open Issues}
\lips

\section{Off-the-Shelf Solutions}
\subsection{Ready-Made Products}
\lips
\subsection{Reusable Components}
\lips
\subsection{Products That Can Be Copied}
\lips

\section{New Problems}
\subsection{Effects on the Current Environment}
\lips
\subsection{Effects on the Installed Systems}
\lips
\subsection{Potential User Problems}
\lips
\subsection{Limitations in the Anticipated Implementation Environment That May
Inhibit the New Product}
\lips
\subsection{Follow-Up Problems}
\lips

\section{Tasks}
\subsection{Project Planning}
\lips
\subsection{Planning of the Development Phases}
\lips

\section{Migration to the New Product}
\subsection{Requirements for Migration to the New Product}
\lips
\subsection{Data That Has to be Modified or Translated for the New System}
\lips

\section{Costs}
\lips
\section{User Documentation and Training}
\subsection{User Documentation Requirements}
\lips
\subsection{Training Requirements}
\lips

\section{Waiting Room}
\lips

\section{Ideas for Solution}
\lips

\newpage{}
\section*{Appendix --- Reflection}

\input{../Reflection.tex}

\begin{enumerate}
  \item What went well while writing this deliverable?
  \begin{itemize}
  \item One thing that went well while writing the SRS was our team’s ability to clearly define
  functional and non-functional requirements based on our project scope.
  We divided the document sections efficiently and maintained consistent formatting and terminology.
  Our discussions on user needs and system goals also helped refine the main features early,
  making later sections like use cases and functional requirements much easier to complete.
  \end{itemize}

  \item What pain points did you experience during this deliverable, and how did
  you resolve them?
  \begin{enumerate}
    \item Discussing and validating requirements with the Professor during the middle of midterm season.\\
    We resolved this by trusting the requirements and feedback from our previous meetings with the Professor, along with course lecture notes, the SRS-Volere template, and other stakeholders to aid us in generating appropriate and relevant requirements.
    \item Reviewing SRS and ensuring all parts are consistent and complete.\\
    This is still unresolved, as many sections in the SRS are inconsistent and still incomplete (like the card game mechanics in Sections 8-9, which appear as complete, but are still being discussed among the team and Project Supervisor Paul Rapoport).\\
    One thing that worked to aid in resolving this pain point, was one of our team members (Ammar) doing an individual review of SRS (before he added all of his sections), to ensure the document is consistent and includes everything important. Upon review, Ammar found several missing subsections, inconsistencies, and suggested improvements, and created a GitHub issue for each one, which was then delegated to the appropriate team member, and resolved by each team member (as written in the created issue).\\
    Another thing that helped was teammate Ammar getting a 3-day extension (which turned into a 4-day extension because of personal circumstances) on his Sections of the SRS. He used this extra time to schedule a meeting with Paul Rapoport and available team members to present our team's SRS, and open it for review to the Professor.\\
    Shortly after the meeting, Supervisor Paul Rapoport sent us an email with his feedback on our SRS, in a written document. This document of Professor Rapoport's review was added to GitHub as an issue (see Project Issue 47).
    Teammate Ammar reviewed the document during his extension time, and tried to resolve what he could within the SRS document. The rest of the open issues that were there prior to the meeting, and the ones presented by Paul Rapoport in his review, were added to Section 18: Open Issues. These issues will eventually be added to the Project Repo as individual Issues, and resolved appropriately.\\
    \item Poor time management of teammate Ammar, last minute rush and missing GitHub skills to merge SRS and HA branch commits to main.\\
    Teammate Ammar has apologized and stated any penalties for late submission should go to him and not other teammates. He also created pull requests for branches SRS and HA, as he was unable to merge these to main himself. To resolve the immediate issue of unmerged commits and branches, teammate Ruida stepped in the day after the deadline to fix branch diversions, resolve merge conflicts and merge teammate Ammar's commit history to main.\\
    This being said, the problem of constant late submission by teammate Ammar still persists, and requires resolution. The three clear options are:\\
    \begin{itemize}
        \item Teammate Ammar improves his time management and meets deadlines in the future.
        \item Teammate Ammar is penalized for the late submission for Deliverable 2 (SRS + HA), and the team continues as is.
        \item Professor Smith, Professor Rapoport and/or TA Chris come up with some other resolution/compromise for this pain point.
    \end{itemize}
  \end{enumerate}
  Written by Ammar Sharbat, 2024-10-15
  \item How many of your requirements were inspired by speaking to your
  client(s) or their proxies (e.g. your peers, stakeholders, potential users)?
\begin{itemize}
\item Most of our system requirements were influenced by feedback from peers and proxy users who represent our target audience—students learning numeral base conversions, additional requirements, such as real-time multiplayer functionality, clear in-game hints, and visual base conversion indicators are added.
\end{itemize}
  \item Which of the courses you have taken, or are currently taking, will help
  your team to be successful with your capstone project.
\begin{itemize}
\item SFWRENG 3A04: Software Design III - Large System Design (Git, designing software architecture, UML diagrams)
\item SFWRENG 4HC3: Human Computer Interfaces (User-Centered Design, Usability Testing)
\item SFWRENG 3RA3: Software Requirements (Git, Github Issues, Writing SRS, Requirements Elicitation)
\item SFWRENG 3S03: Software Testing (Test Case Design, Automated Testing Frameworks)
\item SFWRENG 4C03: Computer Networks and Security (Network Protocols, Security Best Practices)
\item SFWRENG 2AA4: Software Design I (Git, Kanban Board, Object-Oriented Design, Design Patterns)
\end{itemize}

  \item What knowledge and skills will the team collectively need to acquire to
  successfully complete this capstone project?  Examples of possible knowledge
  to acquire include domain specific knowledge from the domain of your
  application, or software engineering knowledge, mechatronics knowledge or
  computer science knowledge.  Skills may be related to technology, or writing,
  or presentation, or team management, etc.  You should look to identify at
  least one item for each team member.
\begin{itemize}

\item Ruida: Frontend Development: Deepening our understanding of React component architecture, animation design, and state management.
\item Alvin: Backend Development: Gaining hands on experience in Node.js, Express, and database management.
\end{itemize}
  \item For each of the knowledge areas and skills identified in the previous
  question, what are at least two approaches to acquiring the knowledge or
  mastering the skill?  Of the identified approaches, which will each team
  member pursue, and why did they make this choice?
	\begin{itemize}
	\item Ruida: Frontend Development (React \& Animation)
	Approaches: (1) Following online React tutorials and official documentation; (2) Experimenting through prototype 	iterations and peer code reviews. I will choose the first approach - follow online react tutorials,  since the official tutorial documentation of react is really straightfoward and easy to learn, lots of code examples are provided.
	\end{itemize}
  \begin{itemize}
  \item Alvin: Backend Development (Node.js, Express, Database Management)
  Approaches: (1) Completing online courses and tutorials on Node.js and Express (2) Reviewing course notes and project repositories from past relevant courses.
  \end{itemize}
\end{enumerate}


\end{document}