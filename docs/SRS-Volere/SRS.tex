% THIS DOCUMENT IS FOLLOWS THE VOLERE TEMPLATE BY Suzanne Robertson and James Robertson
% ONLY THE SECTION HEADINGS ARE PROVIDED
%
% Initial draft from https://github.com/Dieblich/volere
%
% Risks are removed because they are covered by the Hazard Analysis
\documentclass[12pt]{article}

\usepackage{booktabs}
\usepackage{tabularx}
\usepackage{hyperref}
\hypersetup{
    bookmarks=true,         % show bookmarks bar?
      colorlinks=true,      % false: boxed links; true: colored links
    linkcolor=red,          % color of internal links (change box color with linkbordercolor)
    citecolor=green,        % color of links to bibliography
    filecolor=magenta,      % color of file links
    urlcolor=cyan           % color of external links
}

\newcommand{\lips}{\textit{Insert your content here.}}

%% Comments

\usepackage{color}

\newif\ifcomments\commentstrue %displays comments
%\newif\ifcomments\commentsfalse %so that comments do not display

\ifcomments
\newcommand{\authornote}[3]{\textcolor{#1}{[#3 ---#2]}}
\newcommand{\todo}[1]{\textcolor{red}{[TODO: #1]}}
\else
\newcommand{\authornote}[3]{}
\newcommand{\todo}[1]{}
\fi

\newcommand{\wss}[1]{\authornote{magenta}{SS}{#1}} 
\newcommand{\plt}[1]{\authornote{cyan}{TPLT}{#1}} %For explanation of the template
\newcommand{\an}[1]{\authornote{cyan}{Author}{#1}}

%% Common Parts

\newcommand{\progname}{ProgName} % PUT YOUR PROGRAM NAME HERE
\newcommand{\authname}{Team \#, Team Name
\\ Student 1 name
\\ Student 2 name
\\ Student 3 name
\\ Student 4 name} % AUTHOR NAMES                  

\usepackage{hyperref}
    \hypersetup{colorlinks=true, linkcolor=blue, citecolor=blue, filecolor=blue,
                urlcolor=blue, unicode=false}
    \urlstyle{same}
                                


\begin{document}

\title{Software Requirements Specification for \progname: subtitle describing software}
\author{
    Team \#25, The Crazy Four \\[1ex]
    Ruida Chen \\
    Ammar Sharbat \\
    Alvin Qian \\
    Jiaming Li
}
\date{30.09.2025}
	
\maketitle

~\newpage

\pagenumbering{roman}

\tableofcontents

~\newpage

\section*{Revision History}

\begin{table}[hp]
    \caption{Revision History} \label{TblRevisionHistory}
    \begin{tabularx}{\textwidth}{llX}
        \toprule
        \textbf{Date} & \textbf{Developer(s)} & \textbf{Change}\\
        \midrule
        9.29 & Jiaming Li & Purpose of the Project\\
        9.30 & Jiaming Li & Scope of the Product\\
        9.30 & Jiaming Li & FR\\
        \bottomrule
    \end{tabularx}
\end{table}

~\\

~\newpage
\section{Purpose of the Project}
\subsection{User Business}
The purpose of this project is to design and implement an educational card game based on the traditional \textit{Crazy 8s} rule set, but adapted to integrate the \textbf{Dozenal (base-12) number system}.
\begin{itemize}
    \item This project addresses the lack of accessible and engaging tools that introduce alternative number systems in a playful and intuitive way.
    \item By combining a familiar card game mechanic with Dozenal representations and operations, users can gradually build comfort and intuition with the base-12 system.
    \item The primary business value lies in providing a lightweight, fun, and interactive educational tool for students, hobbyists, and anyone interested in number systems beyond decimal.
\end{itemize}
Additionally, the game offers an opportunity to evaluate how gamification can support mathematical learning, and whether abstract concepts (such as base conversions or divisibility in Dozenal) can be effectively taught through play.

\subsection{Goals of the Project}
The goals of this project are:
\begin{itemize}
    \item Educational Integration: Seamlessly incorporate Dozenal concepts (symbols 0--B, factorization, arithmetic) into the gameplay, ensuring that players learn by playing without requiring formal prior knowledge.
    \item Gameplay Design: Deliver a working digital version of \textit{Crazy 8s} that is intuitive, responsive, and enjoyable, while maintaining the familiar flow of the original game and introducing Dozenal-specific mechanics (e.g., matching rules, scoring, or special cards).
    \item Accessibility and Engagement: Create a user-friendly interface that lowers the barrier to learning, accessible for casual users while offering depth for learners who want to explore Dozenal further.
    \item Scalability / Stretch Goals: Explore the potential for extending the system to other educational card or board games, and investigate how different number bases can be taught through similar game mechanics.
\end{itemize}
\section{Stakeholders}
\subsection{Client}
\lips
\subsection{Customer}
\lips
\subsection{Other Stakeholders}
\lips
\subsection{Hands-On Users of the Project}
\lips
\subsection{Personas}
\lips
\subsection{Priorities Assigned to Users}
\lips
\subsection{User Participation}
\lips
\subsection{Maintenance Users and Service Technicians}
\lips

\section{Mandated Constraints}
\subsection{Solution Constraints}
\lips
\subsection{Implementation Environment of the Current System}
\lips
\subsection{Partner or Collaborative Applications}
\lips
\subsection{Off-the-Shelf Software}
\lips
\subsection{Anticipated Workplace Environment}
\lips
\subsection{Schedule Constraints}
\lips
\subsection{Budget Constraints}
\lips
\subsection{Enterprise Constraints}
\lips

\section{Naming Conventions and Terminology}
\subsection{Glossary of All Terms, Including Acronyms, Used by Stakeholders
involved in the Project}
\lips

\section{Relevant Facts And Assumptions}
\subsection{Relevant Facts}
\lips
\subsection{Business Rules}
\lips
\subsection{Assumptions}
\lips

\section{The Scope of the Work}
\subsection{The Current Situation}
\lips
\subsection{The Context of the Work}
\lips
\subsection{Work Partitioning}
\lips
\subsection{Specifying a Business Use Case (BUC)}
\lips

\section{Business Data Model and Data Dictionary}
\subsection{Business Data Model}
\lips
\subsection{Data Dictionary}
\lips


\section{The Scope of the Product}

\subsection{Product Boundary}
The product to be developed is a digital card game application based on the traditional \textit{Crazy 8s} rules, modified to integrate the \textbf{Dozenal (base-12) number system}.
The system boundary includes:
\begin{itemize}
    \item A game engine that supports core Crazy 8s mechanics (drawing, discarding, turn-taking, winning conditions).
    \item Adaptations of rules, card values, and scoring to incorporate Dozenal arithmetic and representations.
    \item A user interface allowing players to interact with the game (play cards, view scores, receive feedback).
    \item Educational prompts or visual aids to help players understand Dozenal concepts.
\end{itemize}
External systems not included in the boundary are: general learning platforms, multiplayer servers beyond basic peer-to-peer/local play, and integrations with unrelated educational tools.

\subsection{Product Use Case Table}
The following table summarizes the primary product use cases (PUCs):

\subsection{Product Use Case Table}
The following table summarizes the primary product use cases (PUCs):

\begin{tabularx}{\textwidth}{|l|X|}
    \hline
    \textbf{PUC \#} & \textbf{Description} \\
    \hline
    PUC-1 & Player starts a new Crazy 8s game with Dozenal-enabled deck. \\
    \hline
    PUC-2 & Player takes a turn by drawing or discarding a card. \\
    \hline
    PUC-3 & System validates whether the played card is legal (same suit, same Dozenal value, or sum = 12). \\
    \hline
    PUC-4 & Player views scores and progress, displayed in both decimal and Dozenal. \\
    \hline
    PUC-5 & System provides hints or explanations to support Dozenal learning. \\
    \hline
    PUC-6 & Game ends when a player wins; final scores are calculated and displayed. \\
    \hline
\end{tabularx}

\subsection{Individual Product Use Cases (PUCs)}

\noindent\textbf{UC1: Start a New Game}
\begin{enumerate}
    \item Player opens the application and selects ``New Game''.
    \item The system initializes a Dozenal-enabled deck (0--B, 10).
    \item The system shuffles the deck.
    \item The system deals cards to each player.
    \item The game state is displayed on the interface.
\end{enumerate}

\noindent\textbf{UC2: Take a Turn}
\begin{enumerate}
    \item The system indicates that it is the player’s turn.
    \item The player chooses either to play a card or to draw from the deck.
    \item If the player chooses a card, the system checks its validity against the discard pile.
    \item If valid, the card is placed onto the discard pile.
    \item If invalid, the system notifies the player and the card remains in the hand.
    \item If the player chooses to draw, the system gives one card from the deck to the player.
\end{enumerate}

\noindent\textbf{UC3: Validate Move}
\begin{enumerate}
    \item The player selects a card to play.
    \item The system retrieves the top card from the discard pile.
    \item The system checks if the move is legal under Dozenal rules:
    \begin{enumerate}
        \item same suit, or
        \item same Dozenal value, or
        \item sum of the two values equals 12 (base-12).
    \end{enumerate}
    \item If the move is legal, the system accepts the card and updates the discard pile.
    \item If not, the system rejects the move and notifies the player.
\end{enumerate}

\noindent\textbf{UC4: View Scores}
\begin{enumerate}
    \item A round ends or the player requests to view scores.
    \item The system calculates points for each player.
    \item The system converts the scores into both decimal and Dozenal.
    \item The system displays the results on screen.
\end{enumerate}

\noindent\textbf{UC5: Provide Hints}
\begin{enumerate}
    \item The player hovers over or selects a card.
    \item The system analyzes the current game state.
    \item The system provides a hint (e.g., ``This card is valid because its sum with the top card equals 12 (base-12).'').
    \item The hint is displayed as text or a visual highlight.
\end{enumerate}

\noindent\textbf{UC6: End Game}
\begin{enumerate}
    \item A player discards their last card.
    \item The system checks if the game-ending condition is satisfied.
    \item If satisfied, the system declares the winner.
    \item The system calculates final scores in both decimal and Dozenal.
    \item The results are displayed on the final game screen.
\end{enumerate}



\section{Functional Requirements}

\subsection{Game Manager}

\begin{enumerate}
    \item \textbf{Start new game:} Game manager shall allow the player to start a new Crazy 8s game with a Dozenal-enabled deck. (FR-1) \\
    \textbf{Rationale:} Players need a way to initialize the game state; starting a new game is the foundation for all other gameplay functions.

    \item \textbf{Turn management:} Game manager shall manage player turns, ensuring that each player either discards a valid card or draws a card. (FR-2) \\
    \textbf{Rationale:} Turn-taking enforces fairness and ensures game flow consistency.

    \item \textbf{Rule validation:} Game manager shall validate that each played card is legal under Dozenal rules. A valid move is defined as either: (a) same suit as the previous card, (b) same Dozenal value as the previous card, or (c) the sum of the two card values equals 12 in Dozenal. (FR-3) \\
    \textbf{Rationale:} Prevents illegal moves, guarantees consistency, and introduces the core educational mechanic of Dozenal arithmetic.

    \item \textbf{Special cards:} Game manager shall implement special card effects (e.g., 8s are wild) while supporting extensions with Dozenal-specific effects. (FR-4) \\
    \textbf{Rationale:} Special cards increase engagement and add flexibility in teaching Dozenal-based rules.

    \item \textbf{End of game:} Game manager shall determine when the game ends (e.g., when a player runs out of cards) and declare the winner. (FR-5) \\
    \textbf{Rationale:} A clear end condition is required for meaningful gameplay and reinforcement of learning objectives.
\end{enumerate}


\subsection{Score Manager}

\begin{enumerate}
    \item \textbf{Calculate score:} Score manager shall calculate points for each round in both decimal and Dozenal. (FR-6) \\
    \textbf{Rationale:} Displaying scores in both systems reinforces learning by encouraging comparison between familiar decimal and new Dozenal formats.

    \item \textbf{Display score:} Score manager shall display both decimal and Dozenal results on the user interface. (FR-7) \\
    \textbf{Rationale:} Visual feedback supports player understanding and helps users internalize Dozenal representations.
\end{enumerate}


\subsection{Education Support}

\begin{enumerate}
    \item \textbf{Hints:} System shall provide hints or explanations when a player performs an action involving Dozenal arithmetic. (FR-8) \\
    \textbf{Rationale:} On-demand guidance lowers the learning curve and supports players with varying levels of familiarity.

    \item \textbf{Highlight valid moves:} System shall visually highlight valid moves based on Dozenal rules. (FR-9) \\
    \textbf{Rationale:} Reduces frustration, ensures players stay engaged, and reinforces Dozenal rules through visual learning.
\end{enumerate}

\section{Look and Feel Requirements}
\subsection{Appearance Requirements}
\lips
\subsection{Style Requirements}
\lips

\section{Usability and Humanity Requirements}
\subsection{Ease of Use Requirements}
\lips
\subsection{Personalization and Internationalization Requirements}
\lips
\subsection{Learning Requirements}
\lips
\subsection{Understandability and Politeness Requirements}
\lips
\subsection{Accessibility Requirements}
\lips

\section{Performance Requirements}
\subsection{Speed and Latency Requirements}
\lips
\subsection{Safety-Critical Requirements}
\lips
\subsection{Precision or Accuracy Requirements}
\lips
\subsection{Robustness or Fault-Tolerance Requirements}
\lips
\subsection{Capacity Requirements}
\lips
\subsection{Scalability or Extensibility Requirements}
\lips
\subsection{Longevity Requirements}
\lips

\section{Operational and Environmental Requirements}
\subsection{Expected Physical Environment}
\lips
\subsection{Wider Environment Requirements}
\lips
\subsection{Requirements for Interfacing with Adjacent Systems}
\lips
\subsection{Productization Requirements}
\lips
\subsection{Release Requirements}
\lips

\section{Maintainability and Support Requirements}
\subsection{Maintenance Requirements}
\lips
\subsection{Supportability Requirements}
\lips
\subsection{Adaptability Requirements}
\lips

\section{Security Requirements}
\subsection{Access Requirements}
\lips
\subsection{Integrity Requirements}
\lips
\subsection{Privacy Requirements}
\lips
\subsection{Audit Requirements}
\lips
\subsection{Immunity Requirements}
\lips

\section{Cultural Requirements}
\subsection{Cultural Requirements}
\lips

\section{Compliance Requirements}
\subsection{Legal Requirements}
\lips
\subsection{Standards Compliance Requirements}
\lips

\section{Open Issues}
\lips

\section{Off-the-Shelf Solutions}
\subsection{Ready-Made Products}
\lips
\subsection{Reusable Components}
\lips
\subsection{Products That Can Be Copied}
\lips

\section{New Problems}
\subsection{Effects on the Current Environment}
\lips
\subsection{Effects on the Installed Systems}
\lips
\subsection{Potential User Problems}
\lips
\subsection{Limitations in the Anticipated Implementation Environment That May
Inhibit the New Product}
\lips
\subsection{Follow-Up Problems}
\lips

\section{Tasks}
\subsection{Project Planning}
\lips
\subsection{Planning of the Development Phases}
\lips

\section{Migration to the New Product}
\subsection{Requirements for Migration to the New Product}
\lips
\subsection{Data That Has to be Modified or Translated for the New System}
\lips

\section{Costs}
\lips
\section{User Documentation and Training}
\subsection{User Documentation Requirements}
\lips
\subsection{Training Requirements}
\lips

\section{Waiting Room}
\lips

\section{Ideas for Solution}
\lips

\newpage{}
\section*{Appendix --- Reflection}

The purpose of reflection questions is to give you a chance to assess your own
learning and that of your group as a whole, and to find ways to improve in the
future. Reflection is an important part of the learning process.  Reflection is
also an essential component of a successful software development process.  

Reflections are most interesting and useful when they're honest, even if the
stories they tell are imperfect. You will be marked based on your depth of
thought and analysis, and not based on the content of the reflections
themselves. Thus, for full marks we encourage you to answer openly and honestly
and to avoid simply writing ``what you think the evaluator wants to hear.''

Please answer the following questions.  Some questions can be answered on the
team level, but where appropriate, each team member should write their own
response:


\begin{enumerate}
  \item What went well while writing this deliverable?
  \begin{itemize}
  \item One thing that went well while writing the SRS was our team’s ability to clearly define
  functional and non-functional requirements based on our project scope.
  We divided the document sections efficiently and maintained consistent formatting and terminology.
  Our discussions on user needs and system goals also helped refine the main features early,
  making later sections like use cases and functional requirements much easier to complete.
  \end{itemize}

  \item What pain points did you experience during this deliverable, and how did
  you resolve them?
  \item How many of your requirements were inspired by speaking to your
  client(s) or their proxies (e.g. your peers, stakeholders, potential users)?
\begin{itemize}
\item Most of our system requirements were influenced by feedback from peers and proxy users who represent our target audience—students learning numeral base conversions, additional requirements, such as real-time multiplayer functionality, clear in-game hints, and visual base conversion indicators are added.
\end{itemize}
  \item Which of the courses you have taken, or are currently taking, will help
  your team to be successful with your capstone project.
\begin{itemize}
\item SFWRENG 3A04: Software Design III - Large System Design (Git, designing software architecture, UML diagrams)
\item SFWRENG 4HC3: Human Computer Interfaces (User-Centered Design, Usability Testing)
\item SFWRENG 3RA3: Software Requirements (Git, Github Issues, Writing SRS, Requirements Elicitation)
\item SFWRENG 3S03: Software Testing (Test Case Design, Automated Testing Frameworks)
\item SFWRENG 4C03: Computer Networks and Security (Network Protocols, Security Best Practices)
\item SFWRENG 2AA4: Software Design I (Git, Kanban Board, Object-Oriented Design, Design Patterns)
\end{itemize}

  \item What knowledge and skills will the team collectively need to acquire to
  successfully complete this capstone project?  Examples of possible knowledge
  to acquire include domain specific knowledge from the domain of your
  application, or software engineering knowledge, mechatronics knowledge or
  computer science knowledge.  Skills may be related to technology, or writing,
  or presentation, or team management, etc.  You should look to identify at
  least one item for each team member.
\begin{itemize}

\item Ruida: Frontend Development: Deepening our understanding of React component architecture, animation design, and state management.
\item Alvin: Backend Development: Gaining hands on experience in Node.js, Express, and database management.
\end{itemize}
  \item For each of the knowledge areas and skills identified in the previous
  question, what are at least two approaches to acquiring the knowledge or
  mastering the skill?  Of the identified approaches, which will each team
  member pursue, and why did they make this choice?
	\begin{itemize}
	\item Ruida: Frontend Development (React \& Animation)
	Approaches: (1) Following online React tutorials and official documentation; (2) Experimenting through prototype 	iterations and peer code reviews. I will choose the first approach - follow online react tutorials,  since the official tutorial documentation of react is really straightfoward and easy to learn, lots of code examples are provided.
	\end{itemize}
  \begin{itemize}
  \item Alvin: Backend Development (Node.js, Express, Database Management)
  Approaches: (1) Completing online courses and tutorials on Node.js and Express (2) Reviewing course notes and project repositories from past relevant courses.
  \end{itemize}
\end{enumerate}


\end{document}