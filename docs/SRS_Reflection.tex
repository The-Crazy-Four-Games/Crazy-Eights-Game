\begin{enumerate}
  \item What went well while writing this deliverable?
  \begin{itemize}
  \item One thing that went well while writing the SRS was our team’s ability to clearly define
  functional and non-functional requirements based on our project scope.
  We divided the document sections efficiently and maintained consistent formatting and terminology.
  Our discussions on user needs and system goals also helped refine the main features early,
  making later sections like use cases and functional requirements much easier to complete.
  \end{itemize}

  \item What pain points did you experience during this deliverable, and how did
  you resolve them?
  \begin{enumerate}
    \item Discussing and validating requirements with the Professor during the middle of midterm season.\\
    We resolved this by trusting the requirements and feedback from our previous meetings with the Professor, along with course lecture notes, the SRS-Volere template, and other stakeholders to aid us in generating appropriate and relevant requirements.
    \item Reviewing SRS and ensuring all parts are consistent and complete.\\
    This is still unresolved, as many sections in the SRS are inconsistent and still incomplete (like the card game mechanics in Sections 8-9, which appear as complete, but are still being discussed among the team and Project Supervisor Paul Rapoport).\\
    One thing that worked to aid in resolving this pain point, was one of our team members (Ammar) doing an individual review of SRS (before he added all of his sections), to ensure the document is consistent and includes everything important. Upon review, Ammar found several missing subsections, inconsistencies, and suggested improvements, and created a GitHub issue for each one, which was then delegated to the appropriate team member, and resolved by each team member (as written in the created issue).\\
    Another thing that helped was teammate Ammar getting a 3-day extension (which turned into a 4-day extension because of personal circumstances) on his Sections of the SRS. He used this extra time to schedule a meeting with Paul Rapoport and available team members to present our team's SRS, and open it for review to the Professor.\\
    Shortly after the meeting, Supervisor Paul Rapoport sent us an email with his feedback on our SRS, in a written document. This document of Professor Rapoport's review was added to GitHub as an issue (see Project Issue 47).
    Teammate Ammar reviewed the document during his extension time, and tried to resolve what he could within the SRS document. The rest of the open issues that were there prior to the meeting, and the ones presented by Paul Rapoport in his review, were added to Section 18: Open Issues. These issues will eventually be added to the Project Repo as individual Issues, and resolved appropriately.\\
    \item Poor time management of teammate Ammar, last minute rush and missing GitHub skills to merge SRS and HA branch commits to main.\\
    Teammate Ammar has apologized and stated any penalties for late submission should go to him and not other teammates. He also created pull requests for branches SRS and HA, as he was unable to merge these to main himself. To resolve the immediate issue of unmerged commits and branches, teammate Ruida stepped in the day after the deadline to fix branch diversions, resolve merge conflicts and merge teammate Ammar's commit history to main.\\
    This being said, the problem of constant late submission by teammate Ammar still persists, and requires resolution. The three clear options are:\\
    \begin{itemize}
        \item Teammate Ammar improves his time management and meets deadlines in the future.
        \item Teammate Ammar is penalized for the late submission for Deliverable 2 (SRS + HA), and the team continues as is.
        \item Professor Smith, Professor Rapoport and/or TA Chris come up with some other resolution/compromise for this pain point.
    \end{itemize}
  \end{enumerate}
  Written by Ammar Sharbat, 2024-10-15
  \item How many of your requirements were inspired by speaking to your
  client(s) or their proxies (e.g. your peers, stakeholders, potential users)?
\begin{itemize}
\item Most of our system requirements were influenced by feedback from peers and proxy users who represent our target audience—students learning numeral base conversions, additional requirements, such as real-time multiplayer functionality, clear in-game hints, and visual base conversion indicators are added.
\end{itemize}
  \item Which of the courses you have taken, or are currently taking, will help
  your team to be successful with your capstone project.
\begin{itemize}
\item SFWRENG 3A04: Software Design III - Large System Design (Git, designing software architecture, UML diagrams)
\item SFWRENG 4HC3: Human Computer Interfaces (User-Centered Design, Usability Testing)
\item SFWRENG 3RA3: Software Requirements (Git, Github Issues, Writing SRS, Requirements Elicitation)
\item SFWRENG 3S03: Software Testing (Test Case Design, Automated Testing Frameworks)
\item SFWRENG 4C03: Computer Networks and Security (Network Protocols, Security Best Practices)
\item SFWRENG 2AA4: Software Design I (Git, Kanban Board, Object-Oriented Design, Design Patterns)
\end{itemize}

  \item What knowledge and skills will the team collectively need to acquire to
  successfully complete this capstone project?  Examples of possible knowledge
  to acquire include domain specific knowledge from the domain of your
  application, or software engineering knowledge, mechatronics knowledge or
  computer science knowledge.  Skills may be related to technology, or writing,
  or presentation, or team management, etc.  You should look to identify at
  least one item for each team member.
\begin{itemize}

\item Ruida: Frontend Development: Deepening our understanding of React component architecture, animation design, and state management.
\item Alvin: Backend Development: Gaining hands on experience in Node.js, Express, and database management.
\end{itemize}
  \item For each of the knowledge areas and skills identified in the previous
  question, what are at least two approaches to acquiring the knowledge or
  mastering the skill?  Of the identified approaches, which will each team
  member pursue, and why did they make this choice?
	\begin{itemize}
	\item Ruida: Frontend Development (React \& Animation)
	Approaches: (1) Following online React tutorials and official documentation; (2) Experimenting through prototype 	iterations and peer code reviews. I will choose the first approach - follow online react tutorials,  since the official tutorial documentation of react is really straightfoward and easy to learn, lots of code examples are provided.
	\end{itemize}
  \begin{itemize}
  \item Alvin: Backend Development (Node.js, Express, Database Management)
  Approaches: (1) Completing online courses and tutorials on Node.js and Express (2) Reviewing course notes and project repositories from past relevant courses.
  \end{itemize}
\end{enumerate}
