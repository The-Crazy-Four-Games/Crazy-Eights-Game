\begin{enumerate}
  \item What went well while writing this deliverable?
  \begin{itemize}
  \item One thing that went well while writing the SRS was our team’s ability to clearly define
  functional and non-functional requirements based on our project scope.
  We divided the document sections efficiently and maintained consistent formatting and terminology.
  Our discussions on user needs and system goals also helped refine the main features early,
  making later sections like use cases and functional requirements much easier to complete.
  \end{itemize}

  \item What pain points did you experience during this deliverable, and how did
  you resolve them?
  \item How many of your requirements were inspired by speaking to your
  client(s) or their proxies (e.g. your peers, stakeholders, potential users)?
\begin{itemize}
\item Most of our system requirements were influenced by feedback from peers and proxy users who represent our target audience—students learning numeral base conversions, additional requirements, such as real-time multiplayer functionality, clear in-game hints, and visual base conversion indicators are added.
\end{itemize}
  \item Which of the courses you have taken, or are currently taking, will help
  your team to be successful with your capstone project.
\begin{itemize}
\item SFWRENG 3A04: Software Design III - Large System Design (Git, designing software architecture, UML diagrams)
\item SFWRENG 4HC3: Human Computer Interfaces (User-Centered Design, Usability Testing)
\item SFWRENG 3RA3: Software Requirements (Git, Github Issues, Writing SRS, Requirements Elicitation)
\item SFWRENG 3S03: Software Testing (Test Case Design, Automated Testing Frameworks)
\item SFWRENG 4C03: Computer Networks and Security (Network Protocols, Security Best Practices)
\item SFWRENG 2AA4: Software Design I (Git, Kanban Board, Object-Oriented Design, Design Patterns)
\end{itemize}

  \item What knowledge and skills will the team collectively need to acquire to
  successfully complete this capstone project?  Examples of possible knowledge
  to acquire include domain specific knowledge from the domain of your
  application, or software engineering knowledge, mechatronics knowledge or
  computer science knowledge.  Skills may be related to technology, or writing,
  or presentation, or team management, etc.  You should look to identify at
  least one item for each team member.
\begin{itemize}

\item Ruida: Frontend Development: Deepening our understanding of React component architecture, animation design, and state management.
\item Alvin: Backend Development: Gaining hands on experience in Node.js, Express, and database management.
\end{itemize}
  \item For each of the knowledge areas and skills identified in the previous
  question, what are at least two approaches to acquiring the knowledge or
  mastering the skill?  Of the identified approaches, which will each team
  member pursue, and why did they make this choice?
	\begin{itemize}
	\item Ruida: Frontend Development (React \& Animation)
	Approaches: (1) Following online React tutorials and official documentation; (2) Experimenting through prototype 	iterations and peer code reviews. I will choose the first approach - follow online react tutorials,  since the official tutorial documentation of react is really straightfoward and easy to learn, lots of code examples are provided.
	\end{itemize}
  \begin{itemize}
  \item Alvin: Backend Development (Node.js, Express, Database Management)
  Approaches: (1) Completing online courses and tutorials on Node.js and Express (2) Reviewing course notes and project repositories from past relevant courses.
  \end{itemize}
\end{enumerate}
